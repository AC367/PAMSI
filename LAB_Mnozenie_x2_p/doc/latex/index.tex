\-Wykonal \-Arkadiusz \-Cyktor, numer indeksu\-: 200367.

\-Program ma za zadanie wczytac dane wejsciowe z pliku, wykonac zadany algorytm, ktory w tym wypadku mnozy wczytane liczby przez 2, a nastepnie porownac otrzymane wyniki z rozwiazaniami zawartymi w pliku wzorcowym. \-Na koniec ma zostac podany czas realizacji algorytmu.\hypertarget{index_etykieta-Wazne-cechy}{}\section{\-Najwazniejsze chechy}\label{index_etykieta-Wazne-cechy}
\-Poprawnosc realizacji poszczegolnych krokow jest kontrolowana przez wyswietlanie stosownych komunikatow.

\-Pomiar czasu dokonywany jest przy uzyciu odpowiednio zaimplementowanej funkcji clock(). \-Wynik pomiaru wswietlany na wyjsciu podawany jest w sekundach, z dokladnoscia do 0.\-01.

\-Mierzony jest tylko czas wykonania algorytmu mnozenia. \-Okres ten jest tak maly, ze zostaje przez program zaokraglony do zera, aby uzyskac inne wartosci nalezy powtarzac mnozenie co najmniej 200000 razy (0.\-01s). 