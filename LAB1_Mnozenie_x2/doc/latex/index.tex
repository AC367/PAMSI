\-Wykonal \-Arkadiusz \-Cyktor, numer indeksu\-: 200367.

\-Program ma za zadanie wczytac dane wejsciowe z pliku, wykonac zadany algorytm (dostepnych jest kilka funkcji, aby wybrac ktoras z nich nelzy usunac komentarz)a nastepnie porownac otrzymane wyniki z rozwiazaniami zawartymi w pliku wzorcowym. \-Na koniec ma zostac podany czas realizacji algorytmu.\hypertarget{index_etykieta-Wazne-cechy}{}\section{\-Najwazniejsze chechy}\label{index_etykieta-Wazne-cechy}
\-Interakcja programu z uzytkownikiem ogranicza sie do podania przez tego drugiego zadanej ilosci powtorzen wykonania algorytmu oraz ilosci liczb, na ktorych ma on pracowac, dlasza czesc wykonywana jest automatycznie i konczy sie wyswietleniem wyniku (w przypadku mnozenia duza ilosc powtorzen moze powodowac wyswietlenie samych zer, poniewaz wartosci sa zbyt duze) oraz czasu trwania calej operacji. \-Dokladniejsze dane, takie jak sredni czas pojedynczego wykonania algorytmu, ilosc liczb oraz liczbe powtorzen algorytmu, zostaja zapisane do pliku csv o nazwie \-Wynik.

\-Poprawnosc realizacji poszczegolnych moze byc kontrolowana przez wyswietlanie stosownych komunikatow, domyslnie sa one \char`\"{}wykomentowane\char`\"{}, jendak nic nie stoi na przeszkodzie, aby przed kompilacja usunac znaczniki komentarzy. \-Takie rozwiazanie powoduje kilka ostrzezen podczas kompilacji, mowiacych o nieuzytych zmiennych, znikaja one po \char`\"{}odkomentowaniu\char`\"{} poszczegolnych petli.

\-Zalaczony plik porownawczy stworzony jest z mysla o algorytmie mnozenia, dla pojedynczej jego realizacji oraz dzialania na 10 wartosciach.

\-Pomiar czasu dokonywany jest przy uzyciu odpowiednio zaimplementowanej funkcji clock(). \-Wynik pomiaru wswietlany na wyjsciu podawany jest w sekundach, z dokladnoscia do 0.\-01.

\-Mierzony jest tylko czas wykonania danego algorytmu. \-Okres ten jest tak maly, ze zostaje przez program zaokraglony do zera, aby uzyskac inne wartosci nalezy powtarzac mnozenie kilka tysiecy razy. 