\documentclass[a4paper,11pt]{report}
\usepackage[T1]{fontenc}
\usepackage[utf8]{inputenc}
\usepackage[polish]{babel}
\usepackage{lmodern}
\usepackage{graphicx}

\title{Implementacja algorytmu rozwiązującego problem plecakowy}
\author{Arkadiusz Cyktor 200367}

\begin{document}
\maketitle


\begin{figure}
  \textbf{Problem plecakowy} można opisać w skrócie, jako dobór takich elementów zbioru, abych ich wartość była jak największa, przy jednoczesnym utrzymaniu ich łącznej wagi poniżej określonego poziomu.
\\
\\Zagadnienie to można rozwiązać na kilka sposobów, ja w swoim programie zdecydowałem się na \textbf{programowanie dynamiczne}. Takie podejście do problemu polega na rzobiciu go na kilka mniejszych podproblemów (potencjalnie łatwiejszych do rozwiązania), a następnie na wykorzystaniu ich rozwiązań do uzyskania nteresującego nas wyniku.
\\W przypdaku problemu plecakowego polega to na rozwiązaniu szeregu zadań, w których rozmiar "plecaka" jest znacznie mniejszy od zadanego.
\\
\\W swoim programie zastosowałem dwuwymiarową tablicę o \emph{N+1} wierszach (gdzie N jest zadaną wielkością "plecaka''), które przedstawiają zwiększający się (od 0 do N) rozmiar problemu, oraz o \emph{X+1} kolumnach, przy czym każda z nich odpowiada jednemu przedmiotowi z zadanego zbioru. Zasada działania algorytmu jest bardzo prosta - dla każdego wiersza obliczane jest optymalne wypełnienie "plecaka" z uwzględnieniem kolejno każdego elementu zbioru, następnie wynik zapisywany jest (przy użyciu klasy \emph{Pole\underline{ }tablicy}) w polu na przecięciu danego wiersza i kolumny. Jeśli któryś z elementów przekracza aktualny rozmiar "plecaka", to w dane pole zostaje przepisana wartość pola poprzedniego. Natomiast w wypadku, gdy element można z powodzeniem umieścić w "plecaku" algorytm cofa się do pola (w tej samej kolumnie), dla którego pojemność plecaka była mniejsza o wagę aktualnie rozpatrywanego elementu i dodaje do jego łącznej wartości wartość obecnego elementu. W ten sposób odwołujemy się do innego najlepszego upakowania elementów, które zostało już wcześniej policzone. Tak uzyskana konfiguracja porównywana jest z poprzednią wartością z wiersza i do aktualnie przetwarzanego pola zapisywana jest ta o większej łącznej wartości upakowanych elementów.
\\Przechodząc w ten sposób wszystkie wiersze i kolumny uzyskamy w ostatnim polu optymalne rozwiązanie głównego problemu.
\\Ponieważ liczba wpisów w tablicy wyników wynosi w*k, (gdzie w to liczba wierszy, a k - kolumn) to złożonością obliczeniową takiego algorytmu jest O(wk).
\end{figure}

\begin{figure}
W programie zostały wprowadzone trzy klasy \emph{Przedmiot}, \emph{Pole\underline{ }tablicy} i \emph{Glowna}. Pierwsza z nich odpowiada za modelowanie przedmiotów, które mogą być umieszczane w "plecaku" - zawiera informacje o ich wadze i wartości, a także indeks, który pozwala na ich identyfikację. Druga klasa odpowiada za realizację pól wspomniajej wcześniej tablicy - znajduje się w niej wektor przechowujący indeksy przedmiotów składających się na optymalne rozwiązanie danego problemu, a także pole typu int przechowujące sumaryczną wartość wspomnianych przedmiotów. Ostatnia klasa przechowuje tablicę rozwiązań oraz dwie globalne zmienne - przechowujące informacje o wielkości "plecaka" oraz ilości elementów.
\end{figure}
\end{document}