\documentclass[a4paper]{book}
\usepackage{makeidx}
\usepackage{natbib}
\usepackage{graphicx}
\usepackage{multicol}
\usepackage{float}
\usepackage{listings}
\usepackage{color}
\usepackage{ifthen}
\usepackage[table]{xcolor}
\usepackage{textcomp}
\usepackage{alltt}
\usepackage{ifpdf}
\ifpdf
\usepackage[pdftex,
            pagebackref=true,
            colorlinks=true,
            linkcolor=blue,
            unicode
           ]{hyperref}
\else
\usepackage[ps2pdf,
            pagebackref=true,
            colorlinks=true,
            linkcolor=blue,
            unicode
           ]{hyperref}
\usepackage{pspicture}
\fi
\usepackage[utf8]{inputenc}
\usepackage{polski}
\usepackage[T1]{fontenc}

\usepackage{mathptmx}
\usepackage[scaled=.90]{helvet}
\usepackage{courier}
\usepackage{sectsty}
\usepackage[titles]{tocloft}
\usepackage{doxygen}
\lstset{language=C++,inputencoding=utf8,basicstyle=\footnotesize,breaklines=true,breakatwhitespace=true,tabsize=8,numbers=left }
\makeindex
\setcounter{tocdepth}{3}
\renewcommand{\footrulewidth}{0.4pt}
\renewcommand{\familydefault}{\sfdefault}
\hfuzz=15pt
\setlength{\emergencystretch}{15pt}
\hbadness=750
\tolerance=750
\begin{document}
\hypersetup{pageanchor=false,citecolor=blue}
\begin{titlepage}
\vspace*{7cm}
\begin{center}
{\Large \-My \-Project }\\
\vspace*{1cm}
{\large \-Wygenerowano przez Doxygen 1.7.6.1}\\
\vspace*{0.5cm}
{\small Wed Apr 23 2014 16:40:17}\\
\end{center}
\end{titlepage}
\clearemptydoublepage
\pagenumbering{roman}
\tableofcontents
\clearemptydoublepage
\pagenumbering{arabic}
\hypersetup{pageanchor=true,citecolor=blue}
\chapter{\-Strona główna}
\label{index}\hypertarget{index}{}\-Wykonal \-Arkadiusz \-Cyktor, numer indeksu\-: 200367.

\-Program ma za zadanie przedstawić realizację algorytmu rozwiązującego problem plecakowy.

\-Do rozwiązania tego zagadnienia posłużyłem się programowaniem dynamicznym. 
\chapter{\-Indeks klas}
\section{\-Lista klas}
\-Tutaj znajdują się klasy, struktury, unie i interfejsy wraz z ich krótkimi opisami\-:\begin{DoxyCompactList}
\item\contentsline{section}{\hyperlink{class_dane}{\-Dane} \\*\-Klasa definiujaca obiekt \hyperlink{class_dane}{\-Dane} }{\pageref{class_dane}}{}
\item\contentsline{section}{\hyperlink{class_kolejka__lista}{\-Kolejka\-\_\-lista} \\*\-Klasa definiujaca kolejke zaimplementowana przy pomocy listy }{\pageref{class_kolejka__lista}}{}
\item\contentsline{section}{\hyperlink{class_kolejka__tablica}{\-Kolejka\-\_\-tablica} \\*\-Klasa definiujaca kolejke zaimplementowana przy pomocy tablicy }{\pageref{class_kolejka__tablica}}{}
\item\contentsline{section}{\hyperlink{class_stos__lista}{\-Stos\-\_\-lista} \\*\-Klasa definiujaca stos zaimplementowany przy pomocy listy }{\pageref{class_stos__lista}}{}
\item\contentsline{section}{\hyperlink{class_stos__tablica}{\-Stos\-\_\-tablica} \\*\-Klasa definiujaca stos zaimplementowana przy pomocy tablicy }{\pageref{class_stos__tablica}}{}
\end{DoxyCompactList}

\chapter{\-Indeks plików}
\section{\-Lista plików}
\-Tutaj znajduje się lista wszystkich plików z ich krótkimi opisami\-:\begin{DoxyCompactList}
\item\contentsline{section}{\hyperlink{drzewo__binarne_8cpp}{drzewo\-\_\-binarne.\-cpp} \\*\-Definicje metod klasy \hyperlink{class_drzewo__binarne}{\-Drzewo\-\_\-binarne} }{\pageref{drzewo__binarne_8cpp}}{}
\item\contentsline{section}{\hyperlink{drzewo__binarne_8hh}{drzewo\-\_\-binarne.\-hh} \\*\-Zawiera definicje klasy \hyperlink{class_drzewo__binarne}{\-Drzewo\-\_\-binarne}, jej metody oraz polecenia załączenia niezbędnych bibliotek }{\pageref{drzewo__binarne_8hh}}{}
\item\contentsline{section}{\hyperlink{funkcje_8cpp}{funkcje.\-cpp} \\*\-Zawiera definicje funkcji uzytych w programie }{\pageref{funkcje_8cpp}}{}
\item\contentsline{section}{\hyperlink{funkcje_8hh}{funkcje.\-hh} \\*\-Zawiera deklaracje funkcji oraz instrukcje zalaczenia bibliotek }{\pageref{funkcje_8hh}}{}
\item\contentsline{section}{\hyperlink{main_8cpp}{main.\-cpp} \\*\-Zawiera definicje glownej funkcji programu }{\pageref{main_8cpp}}{}
\item\contentsline{section}{\hyperlink{tablica__asocjacyjna_8hh}{tablica\-\_\-asocjacyjna.\-hh} \\*\-Zawiera definicję klasy \hyperlink{class_tablica_asocjacyjna}{\-Tablica\-Asocjacyjna}, jej metod oraz instrukcje załączenia poszczególnych bibliotek }{\pageref{tablica__asocjacyjna_8hh}}{}
\item\contentsline{section}{\hyperlink{tablica__asocjacyjna__db_8hh}{tablica\-\_\-asocjacyjna\-\_\-db.\-hh} }{\pageref{tablica__asocjacyjna__db_8hh}}{}
\item\contentsline{section}{\hyperlink{tablica__haszujaca_8cpp}{tablica\-\_\-haszujaca.\-cpp} \\*\-Zawiera definicje metod klasy \hyperlink{class_tablica__haszujaca}{\-Tablica\-\_\-haszujaca} i \hyperlink{class_element}{\-Element} }{\pageref{tablica__haszujaca_8cpp}}{}
\item\contentsline{section}{\hyperlink{tablica__haszujaca_8hh}{tablica\-\_\-haszujaca.\-hh} \\*\-Zawiera definicje klase \hyperlink{class_tablica__haszujaca}{\-Tablica\-\_\-haszujaca} oraz \hyperlink{class_element}{\-Element}, ich metod oraz instrukcje załączenia poszczególnych bibliotek }{\pageref{tablica__haszujaca_8hh}}{}
\end{DoxyCompactList}

\chapter{\-Dokumentacja klas}
\hypertarget{class_drzewo__binarne}{\section{\-Dokumentacja klasy \-Drzewo\-\_\-binarne}
\label{class_drzewo__binarne}\index{\-Drzewo\-\_\-binarne@{\-Drzewo\-\_\-binarne}}
}


\-Klasa definiująca drzewo binarne realizujące implementację tablicy asocjacyjnej.  




{\ttfamily \#include $<$drzewo\-\_\-binarne.\-hh$>$}



\-Diagram współpracy dla \-Drzewo\-\_\-binarne\-:
\subsection*{\-Komponenty}
\begin{DoxyCompactItemize}
\item 
struct \hyperlink{struct_drzewo__binarne_1_1galaz}{galaz}
\begin{DoxyCompactList}\small\item\em \-Struktura realizująca \char`\"{}gałęzie\char`\"{} drzewa binarnego. \end{DoxyCompactList}\end{DoxyCompactItemize}
\subsection*{\-Metody publiczne}
\begin{DoxyCompactItemize}
\item 
\hyperlink{class_drzewo__binarne_a9a57a4711e635e2eb9032b6619ce63fb}{\-Drzewo\-\_\-binarne} ()
\begin{DoxyCompactList}\small\item\em \-Konstruktor. \end{DoxyCompactList}\item 
bool \hyperlink{class_drzewo__binarne_aa9e767a3f55ca4d93e5eab7d8d27633d}{\-Czy\-\_\-puste} () const 
\begin{DoxyCompactList}\small\item\em \-Czy drzewo jest puste. \end{DoxyCompactList}\item 
void \hyperlink{class_drzewo__binarne_a2afc7a05aa63660125f5677c3aa5cfe5}{wyswietl\-\_\-kolejno} ()
\begin{DoxyCompactList}\small\item\em \-Wyświetla zawartość. \end{DoxyCompactList}\item 
void \hyperlink{class_drzewo__binarne_a67ce1de738303b5e88c0771cd4aabe1f}{kolejno} (\hyperlink{struct_drzewo__binarne_1_1galaz}{galaz} $\ast$gal)
\begin{DoxyCompactList}\small\item\em \-Pobiera kolejne wartości. \end{DoxyCompactList}\item 
void \hyperlink{class_drzewo__binarne_a760bf194d205fc3e2f3fbadf15239e27}{dodaj} (string adres, string d)
\begin{DoxyCompactList}\small\item\em \-Dodaje elementy. \end{DoxyCompactList}\item 
void \hyperlink{class_drzewo__binarne_a67361052da854238bd9113e5a88bd184}{usun} (string adres)
\begin{DoxyCompactList}\small\item\em \-Usuwa elementy. \end{DoxyCompactList}\item 
bool \hyperlink{class_drzewo__binarne_abcb8fc1deadbc07c98e21edb46b80cea}{czy\-\_\-jest} (string adres)
\begin{DoxyCompactList}\small\item\em \-Sprawdza istnienie elementu. \end{DoxyCompactList}\end{DoxyCompactItemize}
\subsection*{\-Atrybuty publiczne}
\begin{DoxyCompactItemize}
\item 
\hyperlink{struct_drzewo__binarne_1_1galaz}{galaz} $\ast$ \hyperlink{class_drzewo__binarne_ac338d9a6b97981369e3729d11918ed8d}{korzen}
\item 
int \hyperlink{class_drzewo__binarne_a8c37bbd6488cb39a09f5a0af872941d7}{rozmiar}
\end{DoxyCompactItemize}


\subsection{\-Opis szczegółowy}
\-Klasa ta zawiera wewnętrzną strukturę galaz, która odpowiada za przechowywanie adresu oraz zawartości określonego pola oraz wskaźniki na jego lewą i prawą odnogę. \-Ponadto zawarte są w niej deklarace metod oraz konstruktor. 

\-Definicja w linii 20 pliku drzewo\-\_\-binarne.\-hh.



\subsection{\-Dokumentacja konstruktora i destruktora}
\hypertarget{class_drzewo__binarne_a9a57a4711e635e2eb9032b6619ce63fb}{\index{\-Drzewo\-\_\-binarne@{\-Drzewo\-\_\-binarne}!\-Drzewo\-\_\-binarne@{\-Drzewo\-\_\-binarne}}
\index{\-Drzewo\-\_\-binarne@{\-Drzewo\-\_\-binarne}!Drzewo_binarne@{\-Drzewo\-\_\-binarne}}
\subsubsection[{\-Drzewo\-\_\-binarne}]{\setlength{\rightskip}{0pt plus 5cm}{\bf \-Drzewo\-\_\-binarne\-::\-Drzewo\-\_\-binarne} (
\begin{DoxyParamCaption}
{}
\end{DoxyParamCaption}
)\hspace{0.3cm}{\ttfamily  \mbox{[}inline\mbox{]}}}}\label{class_drzewo__binarne_a9a57a4711e635e2eb9032b6619ce63fb}
\-Konstruktor odpowiedzialny za inicjowanie pól obiektu typu \hyperlink{class_drzewo__binarne}{\-Drzewo\-\_\-binarne}. 

\-Definicja w linii 49 pliku drzewo\-\_\-binarne.\-hh.



\subsection{\-Dokumentacja funkcji składowych}
\hypertarget{class_drzewo__binarne_abcb8fc1deadbc07c98e21edb46b80cea}{\index{\-Drzewo\-\_\-binarne@{\-Drzewo\-\_\-binarne}!czy\-\_\-jest@{czy\-\_\-jest}}
\index{czy\-\_\-jest@{czy\-\_\-jest}!Drzewo_binarne@{\-Drzewo\-\_\-binarne}}
\subsubsection[{czy\-\_\-jest}]{\setlength{\rightskip}{0pt plus 5cm}bool {\bf \-Drzewo\-\_\-binarne\-::czy\-\_\-jest} (
\begin{DoxyParamCaption}
\item[{string}]{adres}
\end{DoxyParamCaption}
)}}\label{class_drzewo__binarne_abcb8fc1deadbc07c98e21edb46b80cea}
\-Metoda ta ma za zadanie sprawdzić czy element o podanym adresie występuje w tablicy asocjacyjnej.


\begin{DoxyParams}[1]{\-Parametry}
\mbox{\tt in}  & {\em adres} & -\/ szukany adres.\\
\hline
\end{DoxyParams}
\begin{DoxyReturn}{\-Zwraca}
true -\/ gdy element o podanym adresie znajduje się w tablicy. false -\/ gdy elementu o podanym adresie nie ma w tablicy. 
\end{DoxyReturn}


\-Definicja w linii 188 pliku drzewo\-\_\-binarne.\-cpp.



\-Oto graf wywołań dla tej funkcji\-:


\hypertarget{class_drzewo__binarne_aa9e767a3f55ca4d93e5eab7d8d27633d}{\index{\-Drzewo\-\_\-binarne@{\-Drzewo\-\_\-binarne}!\-Czy\-\_\-puste@{\-Czy\-\_\-puste}}
\index{\-Czy\-\_\-puste@{\-Czy\-\_\-puste}!Drzewo_binarne@{\-Drzewo\-\_\-binarne}}
\subsubsection[{\-Czy\-\_\-puste}]{\setlength{\rightskip}{0pt plus 5cm}bool {\bf \-Drzewo\-\_\-binarne\-::\-Czy\-\_\-puste} (
\begin{DoxyParamCaption}
{}
\end{DoxyParamCaption}
) const}}\label{class_drzewo__binarne_aa9e767a3f55ca4d93e5eab7d8d27633d}
\-Metoda sprawdzająca czy drzewo binarne jest puste. 

\-Definicja w linii 8 pliku drzewo\-\_\-binarne.\-cpp.



\-Oto graf wywoływań tej funkcji\-:


\hypertarget{class_drzewo__binarne_a760bf194d205fc3e2f3fbadf15239e27}{\index{\-Drzewo\-\_\-binarne@{\-Drzewo\-\_\-binarne}!dodaj@{dodaj}}
\index{dodaj@{dodaj}!Drzewo_binarne@{\-Drzewo\-\_\-binarne}}
\subsubsection[{dodaj}]{\setlength{\rightskip}{0pt plus 5cm}void {\bf \-Drzewo\-\_\-binarne\-::dodaj} (
\begin{DoxyParamCaption}
\item[{string}]{adres, }
\item[{string}]{d}
\end{DoxyParamCaption}
)}}\label{class_drzewo__binarne_a760bf194d205fc3e2f3fbadf15239e27}
\-Metoda ta ma za zadanie dodać element o podanym adresie do tablicy asocjacyjnej. \-Jeśli podany adres jest już zajęty, to wartość do niego przypisana zostanie podmieniona.


\begin{DoxyParams}[1]{\-Parametry}
\mbox{\tt in}  & {\em adres} & -\/ adres, pod którym zostanie zapisany element. \\
\hline
\mbox{\tt in}  & {\em d} & -\/ zmienna typu string, która ma zostać zapisana w tablicy. \\
\hline
\end{DoxyParams}


\-Definicja w linii 13 pliku drzewo\-\_\-binarne.\-cpp.



\-Oto graf wywołań dla tej funkcji\-:


\hypertarget{class_drzewo__binarne_a67ce1de738303b5e88c0771cd4aabe1f}{\index{\-Drzewo\-\_\-binarne@{\-Drzewo\-\_\-binarne}!kolejno@{kolejno}}
\index{kolejno@{kolejno}!Drzewo_binarne@{\-Drzewo\-\_\-binarne}}
\subsubsection[{kolejno}]{\setlength{\rightskip}{0pt plus 5cm}void {\bf \-Drzewo\-\_\-binarne\-::kolejno} (
\begin{DoxyParamCaption}
\item[{{\bf galaz} $\ast$}]{gal}
\end{DoxyParamCaption}
)}}\label{class_drzewo__binarne_a67ce1de738303b5e88c0771cd4aabe1f}
\-Metoda ta ma za zadanie pobrać kolejne wartości z drzewa.


\begin{DoxyParams}[1]{\-Parametry}
\mbox{\tt in}  & {\em gal} & -\/ gałąź, od której zaczyna się pobieranie. \\
\hline
\end{DoxyParams}


\-Definicja w linii 175 pliku drzewo\-\_\-binarne.\-cpp.



\-Oto graf wywoływań tej funkcji\-:


\hypertarget{class_drzewo__binarne_a67361052da854238bd9113e5a88bd184}{\index{\-Drzewo\-\_\-binarne@{\-Drzewo\-\_\-binarne}!usun@{usun}}
\index{usun@{usun}!Drzewo_binarne@{\-Drzewo\-\_\-binarne}}
\subsubsection[{usun}]{\setlength{\rightskip}{0pt plus 5cm}void {\bf \-Drzewo\-\_\-binarne\-::usun} (
\begin{DoxyParamCaption}
\item[{string}]{adres}
\end{DoxyParamCaption}
)}}\label{class_drzewo__binarne_a67361052da854238bd9113e5a88bd184}
\-Metoda ta ma za zadanie usunąć element o podanym adresie z tablicy asocjacyjnej. \-Pod warunkiem, że dany adres istnieje.


\begin{DoxyParams}[1]{\-Parametry}
\mbox{\tt in}  & {\em adres} & -\/ adres elementu, który ma zostać usunięty. \\
\hline
\end{DoxyParams}


\-Definicja w linii 50 pliku drzewo\-\_\-binarne.\-cpp.



\-Oto graf wywołań dla tej funkcji\-:


\hypertarget{class_drzewo__binarne_a2afc7a05aa63660125f5677c3aa5cfe5}{\index{\-Drzewo\-\_\-binarne@{\-Drzewo\-\_\-binarne}!wyswietl\-\_\-kolejno@{wyswietl\-\_\-kolejno}}
\index{wyswietl\-\_\-kolejno@{wyswietl\-\_\-kolejno}!Drzewo_binarne@{\-Drzewo\-\_\-binarne}}
\subsubsection[{wyswietl\-\_\-kolejno}]{\setlength{\rightskip}{0pt plus 5cm}void {\bf \-Drzewo\-\_\-binarne\-::wyswietl\-\_\-kolejno} (
\begin{DoxyParamCaption}
{}
\end{DoxyParamCaption}
)}}\label{class_drzewo__binarne_a2afc7a05aa63660125f5677c3aa5cfe5}
\-Metoda odpowiedzialna za wyświetlenie zawartości drzewa binarnego. 

\-Definicja w linii 170 pliku drzewo\-\_\-binarne.\-cpp.



\-Oto graf wywołań dla tej funkcji\-:




\subsection{\-Dokumentacja atrybutów składowych}
\hypertarget{class_drzewo__binarne_ac338d9a6b97981369e3729d11918ed8d}{\index{\-Drzewo\-\_\-binarne@{\-Drzewo\-\_\-binarne}!korzen@{korzen}}
\index{korzen@{korzen}!Drzewo_binarne@{\-Drzewo\-\_\-binarne}}
\subsubsection[{korzen}]{\setlength{\rightskip}{0pt plus 5cm}{\bf galaz}$\ast$ {\bf \-Drzewo\-\_\-binarne\-::korzen}}}\label{class_drzewo__binarne_ac338d9a6b97981369e3729d11918ed8d}


\-Definicja w linii 38 pliku drzewo\-\_\-binarne.\-hh.

\hypertarget{class_drzewo__binarne_a8c37bbd6488cb39a09f5a0af872941d7}{\index{\-Drzewo\-\_\-binarne@{\-Drzewo\-\_\-binarne}!rozmiar@{rozmiar}}
\index{rozmiar@{rozmiar}!Drzewo_binarne@{\-Drzewo\-\_\-binarne}}
\subsubsection[{rozmiar}]{\setlength{\rightskip}{0pt plus 5cm}int {\bf \-Drzewo\-\_\-binarne\-::rozmiar}}}\label{class_drzewo__binarne_a8c37bbd6488cb39a09f5a0af872941d7}


\-Definicja w linii 42 pliku drzewo\-\_\-binarne.\-hh.



\-Dokumentacja dla tej klasy została wygenerowana z plików\-:\begin{DoxyCompactItemize}
\item 
\hyperlink{drzewo__binarne_8hh}{drzewo\-\_\-binarne.\-hh}\item 
\hyperlink{drzewo__binarne_8cpp}{drzewo\-\_\-binarne.\-cpp}\end{DoxyCompactItemize}

\hypertarget{class_element}{\section{\-Dokumentacja klasy \-Element}
\label{class_element}\index{\-Element@{\-Element}}
}


\-Klasa definiująca element umieszczany w tablicy asocjacyjnej.  




{\ttfamily \#include $<$tablica\-\_\-haszujaca.\-hh$>$}



\-Diagram współpracy dla \-Element\-:
\subsection*{\-Metody publiczne}
\begin{DoxyCompactItemize}
\item 
\hyperlink{class_element_ab0d0e20be9a36ae676202db753faeec9}{\-Element} ()
\end{DoxyCompactItemize}
\subsection*{\-Atrybuty publiczne}
\begin{DoxyCompactItemize}
\item 
string \hyperlink{class_element_a3c69cfe4adca29bca0ca6eac06298366}{adres}
\item 
bool \hyperlink{class_element_ae689c2af2c94f3b9fb1c07e89358b3a0}{zajety}
\item 
\hyperlink{class_element}{\-Element} $\ast$ \hyperlink{class_element_a04ef972f58a1777f38b54df88df2fc12}{nastepny}
\end{DoxyCompactItemize}
\subsection*{\-Przyjaciele}
\begin{DoxyCompactItemize}
\item 
class \hyperlink{class_element_a7ea87400dc1aab2d9cdf44457d24cd2a}{\-Tablica\-\_\-haszujaca}
\item 
ostream \& \hyperlink{class_element_a5fbcf8eef969e643764a485bc6f3a100}{operator$<$$<$} (ostream \&ostr, \hyperlink{class_element}{\-Element} \&elem)
\begin{DoxyCompactList}\small\item\em \-Przeładowanie operatora wyjścia. \end{DoxyCompactList}\end{DoxyCompactItemize}


\subsection{\-Opis szczegółowy}
\-Klasa ta zawiera wskaźnik na następny element oraz pola przechowujące adres i dane elementu. \-Jest tu również pole, \char`\"{}zajety\char`\"{} decydujące o możliwości nadpisania danego elementu. \-Ponadto zawarte są w niej deklaracje metod oraz konstruktor. 

\-Definicja w linii 22 pliku tablica\-\_\-haszujaca.\-hh.



\subsection{\-Dokumentacja konstruktora i destruktora}
\hypertarget{class_element_ab0d0e20be9a36ae676202db753faeec9}{\index{\-Element@{\-Element}!\-Element@{\-Element}}
\index{\-Element@{\-Element}!Element@{\-Element}}
\subsubsection[{\-Element}]{\setlength{\rightskip}{0pt plus 5cm}{\bf \-Element\-::\-Element} (
\begin{DoxyParamCaption}
{}
\end{DoxyParamCaption}
)\hspace{0.3cm}{\ttfamily  \mbox{[}inline\mbox{]}}}}\label{class_element_ab0d0e20be9a36ae676202db753faeec9}


\-Definicja w linii 32 pliku tablica\-\_\-haszujaca.\-hh.



\subsection{\-Dokumentacja przyjaciół i funkcji związanych}
\hypertarget{class_element_a5fbcf8eef969e643764a485bc6f3a100}{\index{\-Element@{\-Element}!operator$<$$<$@{operator$<$$<$}}
\index{operator$<$$<$@{operator$<$$<$}!Element@{\-Element}}
\subsubsection[{operator$<$$<$}]{\setlength{\rightskip}{0pt plus 5cm}ostream\& operator$<$$<$ (
\begin{DoxyParamCaption}
\item[{ostream \&}]{ostr, }
\item[{{\bf \-Element} \&}]{elem}
\end{DoxyParamCaption}
)\hspace{0.3cm}{\ttfamily  \mbox{[}friend\mbox{]}}}}\label{class_element_a5fbcf8eef969e643764a485bc6f3a100}
\-Ten przeładowany operator wyjścia odpowiada za podwanie na standardowe wyjście zawartości obiektu klasy \hyperlink{class_element}{\-Element}.


\begin{DoxyParams}[1]{\-Parametry}
\mbox{\tt out}  & {\em ostr} & -\/ strumień wyjściowy. \\
\hline
\mbox{\tt in}  & {\em elem} & -\/ drukowany obiekt. \\
\hline
\end{DoxyParams}


\-Definicja w linii 65 pliku tablica\-\_\-haszujaca.\-cpp.

\hypertarget{class_element_a7ea87400dc1aab2d9cdf44457d24cd2a}{\index{\-Element@{\-Element}!\-Tablica\-\_\-haszujaca@{\-Tablica\-\_\-haszujaca}}
\index{\-Tablica\-\_\-haszujaca@{\-Tablica\-\_\-haszujaca}!Element@{\-Element}}
\subsubsection[{\-Tablica\-\_\-haszujaca}]{\setlength{\rightskip}{0pt plus 5cm}friend class {\bf \-Tablica\-\_\-haszujaca}\hspace{0.3cm}{\ttfamily  \mbox{[}friend\mbox{]}}}}\label{class_element_a7ea87400dc1aab2d9cdf44457d24cd2a}


\-Definicja w linii 37 pliku tablica\-\_\-haszujaca.\-hh.



\subsection{\-Dokumentacja atrybutów składowych}
\hypertarget{class_element_a3c69cfe4adca29bca0ca6eac06298366}{\index{\-Element@{\-Element}!adres@{adres}}
\index{adres@{adres}!Element@{\-Element}}
\subsubsection[{adres}]{\setlength{\rightskip}{0pt plus 5cm}string {\bf \-Element\-::adres}}}\label{class_element_a3c69cfe4adca29bca0ca6eac06298366}


\-Definicja w linii 28 pliku tablica\-\_\-haszujaca.\-hh.

\hypertarget{class_element_a04ef972f58a1777f38b54df88df2fc12}{\index{\-Element@{\-Element}!nastepny@{nastepny}}
\index{nastepny@{nastepny}!Element@{\-Element}}
\subsubsection[{nastepny}]{\setlength{\rightskip}{0pt plus 5cm}{\bf \-Element}$\ast$ {\bf \-Element\-::nastepny}}}\label{class_element_a04ef972f58a1777f38b54df88df2fc12}


\-Definicja w linii 30 pliku tablica\-\_\-haszujaca.\-hh.

\hypertarget{class_element_ae689c2af2c94f3b9fb1c07e89358b3a0}{\index{\-Element@{\-Element}!zajety@{zajety}}
\index{zajety@{zajety}!Element@{\-Element}}
\subsubsection[{zajety}]{\setlength{\rightskip}{0pt plus 5cm}bool {\bf \-Element\-::zajety}}}\label{class_element_ae689c2af2c94f3b9fb1c07e89358b3a0}


\-Definicja w linii 29 pliku tablica\-\_\-haszujaca.\-hh.



\-Dokumentacja dla tej klasy została wygenerowana z pliku\-:\begin{DoxyCompactItemize}
\item 
\hyperlink{tablica__haszujaca_8hh}{tablica\-\_\-haszujaca.\-hh}\end{DoxyCompactItemize}

\hypertarget{struct_drzewo__binarne_1_1galaz}{\section{\-Dokumentacja struktury \-Drzewo\-\_\-binarne\-:\-:galaz}
\label{struct_drzewo__binarne_1_1galaz}\index{\-Drzewo\-\_\-binarne\-::galaz@{\-Drzewo\-\_\-binarne\-::galaz}}
}


\-Struktura realizująca \char`\"{}gałęzie\char`\"{} drzewa binarnego.  




{\ttfamily \#include $<$drzewo\-\_\-binarne.\-hh$>$}



\-Diagram współpracy dla \-Drzewo\-\_\-binarne\-:\-:galaz\-:
\subsection*{\-Atrybuty publiczne}
\begin{DoxyCompactItemize}
\item 
\hyperlink{struct_drzewo__binarne_1_1galaz}{galaz} $\ast$ \hyperlink{struct_drzewo__binarne_1_1galaz_a7ed73bf073476a6b199013ba4caf9b5c}{lewa}
\item 
\hyperlink{struct_drzewo__binarne_1_1galaz}{galaz} $\ast$ \hyperlink{struct_drzewo__binarne_1_1galaz_a185391a82b9a9e74e34a35208f02feb2}{prawa}
\item 
string \hyperlink{struct_drzewo__binarne_1_1galaz_a187601bc8d422803a2ef857e8104b0a3}{adres}
\item 
string \hyperlink{struct_drzewo__binarne_1_1galaz_acb0e8aa37cf38024b7bb98aca46e7f55}{dane}
\end{DoxyCompactItemize}


\subsection{\-Opis szczegółowy}
\-Struktura ta zawiera wskaźniki na lewą i prawą odnogę oraz pola przechowujące informacje o adresie i zawartości gałęzi. 

\-Definicja w linii 30 pliku drzewo\-\_\-binarne.\-hh.



\subsection{\-Dokumentacja atrybutów składowych}
\hypertarget{struct_drzewo__binarne_1_1galaz_a187601bc8d422803a2ef857e8104b0a3}{\index{\-Drzewo\-\_\-binarne\-::galaz@{\-Drzewo\-\_\-binarne\-::galaz}!adres@{adres}}
\index{adres@{adres}!Drzewo_binarne::galaz@{\-Drzewo\-\_\-binarne\-::galaz}}
\subsubsection[{adres}]{\setlength{\rightskip}{0pt plus 5cm}string {\bf \-Drzewo\-\_\-binarne\-::galaz\-::adres}}}\label{struct_drzewo__binarne_1_1galaz_a187601bc8d422803a2ef857e8104b0a3}


\-Definicja w linii 34 pliku drzewo\-\_\-binarne.\-hh.

\hypertarget{struct_drzewo__binarne_1_1galaz_acb0e8aa37cf38024b7bb98aca46e7f55}{\index{\-Drzewo\-\_\-binarne\-::galaz@{\-Drzewo\-\_\-binarne\-::galaz}!dane@{dane}}
\index{dane@{dane}!Drzewo_binarne::galaz@{\-Drzewo\-\_\-binarne\-::galaz}}
\subsubsection[{dane}]{\setlength{\rightskip}{0pt plus 5cm}string {\bf \-Drzewo\-\_\-binarne\-::galaz\-::dane}}}\label{struct_drzewo__binarne_1_1galaz_acb0e8aa37cf38024b7bb98aca46e7f55}


\-Definicja w linii 35 pliku drzewo\-\_\-binarne.\-hh.

\hypertarget{struct_drzewo__binarne_1_1galaz_a7ed73bf073476a6b199013ba4caf9b5c}{\index{\-Drzewo\-\_\-binarne\-::galaz@{\-Drzewo\-\_\-binarne\-::galaz}!lewa@{lewa}}
\index{lewa@{lewa}!Drzewo_binarne::galaz@{\-Drzewo\-\_\-binarne\-::galaz}}
\subsubsection[{lewa}]{\setlength{\rightskip}{0pt plus 5cm}{\bf galaz}$\ast$ {\bf \-Drzewo\-\_\-binarne\-::galaz\-::lewa}}}\label{struct_drzewo__binarne_1_1galaz_a7ed73bf073476a6b199013ba4caf9b5c}


\-Definicja w linii 32 pliku drzewo\-\_\-binarne.\-hh.

\hypertarget{struct_drzewo__binarne_1_1galaz_a185391a82b9a9e74e34a35208f02feb2}{\index{\-Drzewo\-\_\-binarne\-::galaz@{\-Drzewo\-\_\-binarne\-::galaz}!prawa@{prawa}}
\index{prawa@{prawa}!Drzewo_binarne::galaz@{\-Drzewo\-\_\-binarne\-::galaz}}
\subsubsection[{prawa}]{\setlength{\rightskip}{0pt plus 5cm}{\bf galaz}$\ast$ {\bf \-Drzewo\-\_\-binarne\-::galaz\-::prawa}}}\label{struct_drzewo__binarne_1_1galaz_a185391a82b9a9e74e34a35208f02feb2}


\-Definicja w linii 33 pliku drzewo\-\_\-binarne.\-hh.



\-Dokumentacja dla tej struktury została wygenerowana z pliku\-:\begin{DoxyCompactItemize}
\item 
\hyperlink{drzewo__binarne_8hh}{drzewo\-\_\-binarne.\-hh}\end{DoxyCompactItemize}

\hypertarget{class_tablica__haszujaca}{\section{\-Dokumentacja klasy \-Tablica\-\_\-haszujaca}
\label{class_tablica__haszujaca}\index{\-Tablica\-\_\-haszujaca@{\-Tablica\-\_\-haszujaca}}
}


\-Klasa definiująca tablicę haszującą realizującą implementację tablicy asocjacyjnej.  




{\ttfamily \#include $<$tablica\-\_\-haszujaca.\-hh$>$}

\subsection*{\-Metody publiczne}
\begin{DoxyCompactItemize}
\item 
\hyperlink{class_tablica__haszujaca_a10787b7333fcb7fe360ba89854b3351f}{\-Tablica\-\_\-haszujaca} (int il)
\begin{DoxyCompactList}\small\item\em \-Kostruktor. \end{DoxyCompactList}\item 
\hyperlink{class_element}{\-Element} $\ast$ \hyperlink{class_tablica__haszujaca_a7d0fee5b9be90c0403d74821706eaa65}{operator\mbox{[}$\,$\mbox{]}} (string adres)
\begin{DoxyCompactList}\small\item\em \-Przeładowanie operatora \mbox{[}\mbox{]}. \end{DoxyCompactList}\item 
void \hyperlink{class_tablica__haszujaca_af7f6a8c598a22328c585c849754492b7}{dodaj} (string adres, string dane)
\begin{DoxyCompactList}\small\item\em \-Dodaje elementy. \end{DoxyCompactList}\item 
void \hyperlink{class_tablica__haszujaca_afc4feffb08feba48c17f72af2f43bf03}{usun} (string adres)
\begin{DoxyCompactList}\small\item\em \-Usuwa elementy. \end{DoxyCompactList}\item 
\hyperlink{class_element}{\-Element} $\ast$ \hyperlink{class_tablica__haszujaca_ac166065397ac53014b3453543c800337}{znajdz} (string adres)
\begin{DoxyCompactList}\small\item\em \-Znajduje elementy. \end{DoxyCompactList}\item 
int \hyperlink{class_tablica__haszujaca_ac8dbd2993ea46777b616a8b3a0fc20a3}{haszuj} (string adres)
\begin{DoxyCompactList}\small\item\em \-Tworzy klucz. \end{DoxyCompactList}\end{DoxyCompactItemize}
\subsection*{\-Przyjaciele}
\begin{DoxyCompactItemize}
\item 
class \hyperlink{class_tablica__haszujaca_a016b821f88c7c0a2de1451c175cefbf9}{\-Element}
\end{DoxyCompactItemize}


\subsection{\-Opis szczegółowy}
\-Klasa ta zawiera wskaźnik na obiekt klasy \hyperlink{class_element}{\-Element}, pole przechowujące informację o rozmairze oraz deklaracje konstruktora i metod. 

\-Definicja w linii 57 pliku tablica\-\_\-haszujaca.\-hh.



\subsection{\-Dokumentacja konstruktora i destruktora}
\hypertarget{class_tablica__haszujaca_a10787b7333fcb7fe360ba89854b3351f}{\index{\-Tablica\-\_\-haszujaca@{\-Tablica\-\_\-haszujaca}!\-Tablica\-\_\-haszujaca@{\-Tablica\-\_\-haszujaca}}
\index{\-Tablica\-\_\-haszujaca@{\-Tablica\-\_\-haszujaca}!Tablica_haszujaca@{\-Tablica\-\_\-haszujaca}}
\subsubsection[{\-Tablica\-\_\-haszujaca}]{\setlength{\rightskip}{0pt plus 5cm}{\bf \-Tablica\-\_\-haszujaca\-::\-Tablica\-\_\-haszujaca} (
\begin{DoxyParamCaption}
\item[{int}]{il}
\end{DoxyParamCaption}
)}}\label{class_tablica__haszujaca_a10787b7333fcb7fe360ba89854b3351f}
\-Inicjuje pola obiektu klasy \hyperlink{class_tablica__haszujaca}{\-Tablica\-\_\-haszujaca}. 

\-Definicja w linii 8 pliku tablica\-\_\-haszujaca.\-cpp.



\subsection{\-Dokumentacja funkcji składowych}
\hypertarget{class_tablica__haszujaca_af7f6a8c598a22328c585c849754492b7}{\index{\-Tablica\-\_\-haszujaca@{\-Tablica\-\_\-haszujaca}!dodaj@{dodaj}}
\index{dodaj@{dodaj}!Tablica_haszujaca@{\-Tablica\-\_\-haszujaca}}
\subsubsection[{dodaj}]{\setlength{\rightskip}{0pt plus 5cm}void {\bf \-Tablica\-\_\-haszujaca\-::dodaj} (
\begin{DoxyParamCaption}
\item[{string}]{adres, }
\item[{string}]{dane}
\end{DoxyParamCaption}
)}}\label{class_tablica__haszujaca_af7f6a8c598a22328c585c849754492b7}
\-Ta metoda ma za zadanie dodać element o określonym adresie do tablicy asocjacyjnej. \-W przypaku, gdy element o podanym adresie (lub kluczu) już istnieje następuje dopisanie kolejnej wartości do listy przypisanej danemu adresowi.


\begin{DoxyParams}[1]{\-Parametry}
\mbox{\tt in}  & {\em adres} & -\/ adres pod którym ma zostać umieszczony element. \\
\hline
\mbox{\tt in}  & {\em dane} & -\/ wartość, która ma zostać umieszczona w tablicy. \\
\hline
\end{DoxyParams}


\-Definicja w linii 14 pliku tablica\-\_\-haszujaca.\-cpp.



\-Oto graf wywołań dla tej funkcji\-:


\hypertarget{class_tablica__haszujaca_ac8dbd2993ea46777b616a8b3a0fc20a3}{\index{\-Tablica\-\_\-haszujaca@{\-Tablica\-\_\-haszujaca}!haszuj@{haszuj}}
\index{haszuj@{haszuj}!Tablica_haszujaca@{\-Tablica\-\_\-haszujaca}}
\subsubsection[{haszuj}]{\setlength{\rightskip}{0pt plus 5cm}int {\bf \-Tablica\-\_\-haszujaca\-::haszuj} (
\begin{DoxyParamCaption}
\item[{string}]{adres}
\end{DoxyParamCaption}
)}}\label{class_tablica__haszujaca_ac8dbd2993ea46777b616a8b3a0fc20a3}
\-Ta metoda tworzy klucz (indeks) na podstawie adresu elementu.


\begin{DoxyParams}[1]{\-Parametry}
\mbox{\tt in}  & {\em adres} & -\/ adres elementu. \\
\hline
\end{DoxyParams}


\-Definicja w linii 44 pliku tablica\-\_\-haszujaca.\-cpp.



\-Oto graf wywoływań tej funkcji\-:


\hypertarget{class_tablica__haszujaca_a7d0fee5b9be90c0403d74821706eaa65}{\index{\-Tablica\-\_\-haszujaca@{\-Tablica\-\_\-haszujaca}!operator\mbox{[}$\,$\mbox{]}@{operator[]}}
\index{operator\mbox{[}$\,$\mbox{]}@{operator[]}!Tablica_haszujaca@{\-Tablica\-\_\-haszujaca}}
\subsubsection[{operator[]}]{\setlength{\rightskip}{0pt plus 5cm}{\bf \-Element} $\ast$ \-Tablica\-\_\-haszujaca\-::operator\mbox{[}$\,$\mbox{]} (
\begin{DoxyParamCaption}
\item[{string}]{adres}
\end{DoxyParamCaption}
)}}\label{class_tablica__haszujaca_a7d0fee5b9be90c0403d74821706eaa65}
\-Ten przeładowany operator odpowiada za zwrócenie obiektu klasy \hyperlink{class_element}{\-Element} o podanym adresie.


\begin{DoxyParams}[1]{\-Parametry}
\mbox{\tt in}  & {\em adres} & -\/ adres wywoływanego obiektu. \\
\hline
\end{DoxyParams}


\-Definicja w linii 59 pliku tablica\-\_\-haszujaca.\-cpp.



\-Oto graf wywołań dla tej funkcji\-:


\hypertarget{class_tablica__haszujaca_afc4feffb08feba48c17f72af2f43bf03}{\index{\-Tablica\-\_\-haszujaca@{\-Tablica\-\_\-haszujaca}!usun@{usun}}
\index{usun@{usun}!Tablica_haszujaca@{\-Tablica\-\_\-haszujaca}}
\subsubsection[{usun}]{\setlength{\rightskip}{0pt plus 5cm}void {\bf \-Tablica\-\_\-haszujaca\-::usun} (
\begin{DoxyParamCaption}
\item[{string}]{adres}
\end{DoxyParamCaption}
)}}\label{class_tablica__haszujaca_afc4feffb08feba48c17f72af2f43bf03}
\-Ta metoda ma zmienia zawartość pola \char`\"{}zajety\char`\"{} elementu tablicy, przez co umożliwia jego nadpisanie.


\begin{DoxyParams}[1]{\-Parametry}
\mbox{\tt in}  & {\em adres} & -\/ adres usuwanego elementu. \\
\hline
\end{DoxyParams}


\-Definicja w linii 39 pliku tablica\-\_\-haszujaca.\-cpp.



\-Oto graf wywołań dla tej funkcji\-:


\hypertarget{class_tablica__haszujaca_ac166065397ac53014b3453543c800337}{\index{\-Tablica\-\_\-haszujaca@{\-Tablica\-\_\-haszujaca}!znajdz@{znajdz}}
\index{znajdz@{znajdz}!Tablica_haszujaca@{\-Tablica\-\_\-haszujaca}}
\subsubsection[{znajdz}]{\setlength{\rightskip}{0pt plus 5cm}{\bf \-Element} $\ast$ {\bf \-Tablica\-\_\-haszujaca\-::znajdz} (
\begin{DoxyParamCaption}
\item[{string}]{adres}
\end{DoxyParamCaption}
)}}\label{class_tablica__haszujaca_ac166065397ac53014b3453543c800337}
\-Ta metoda ma za zadanie znaleźć i zwrócić obiekt o podanym adresie.


\begin{DoxyParams}[1]{\-Parametry}
\mbox{\tt in}  & {\em adres} & -\/ adres szukanego elementu. \\
\hline
\end{DoxyParams}


\-Definicja w linii 53 pliku tablica\-\_\-haszujaca.\-cpp.



\-Oto graf wywołań dla tej funkcji\-:




\-Oto graf wywoływań tej funkcji\-:




\subsection{\-Dokumentacja przyjaciół i funkcji związanych}
\hypertarget{class_tablica__haszujaca_a016b821f88c7c0a2de1451c175cefbf9}{\index{\-Tablica\-\_\-haszujaca@{\-Tablica\-\_\-haszujaca}!\-Element@{\-Element}}
\index{\-Element@{\-Element}!Tablica_haszujaca@{\-Tablica\-\_\-haszujaca}}
\subsubsection[{\-Element}]{\setlength{\rightskip}{0pt plus 5cm}friend class {\bf \-Element}\hspace{0.3cm}{\ttfamily  \mbox{[}friend\mbox{]}}}}\label{class_tablica__haszujaca_a016b821f88c7c0a2de1451c175cefbf9}


\-Definicja w linii 65 pliku tablica\-\_\-haszujaca.\-hh.



\-Dokumentacja dla tej klasy została wygenerowana z plików\-:\begin{DoxyCompactItemize}
\item 
\hyperlink{tablica__haszujaca_8hh}{tablica\-\_\-haszujaca.\-hh}\item 
\hyperlink{tablica__haszujaca_8cpp}{tablica\-\_\-haszujaca.\-cpp}\end{DoxyCompactItemize}

\hypertarget{class_tablica_asocjacyjna}{\section{\-Dokumentacja szablonu klasy \-Tablica\-Asocjacyjna$<$ \-T $>$}
\label{class_tablica_asocjacyjna}\index{\-Tablica\-Asocjacyjna$<$ T $>$@{\-Tablica\-Asocjacyjna$<$ T $>$}}
}


\-Klasa definiujaca tablice asocjacyjna zaimplementowana przy pomocy wektora.  




{\ttfamily \#include $<$tablica\-\_\-asocjacyjna.\-hh$>$}

\subsection*{\-Komponenty}
\begin{DoxyCompactItemize}
\item 
struct {\bfseries \-\_\-\-Dane}
\end{DoxyCompactItemize}
\subsection*{\-Metody publiczne}
\begin{DoxyCompactItemize}
\item 
long \hyperlink{class_tablica_asocjacyjna_aa73ffd094410a190b69b0dbe294f0489}{\-Rozmiar} ()
\begin{DoxyCompactList}\small\item\em \-Metoda zwracająca informację o ilości danych w tablicy. \end{DoxyCompactList}\item 
bool \hyperlink{class_tablica_asocjacyjna_a21bae8b633b7bde70eab9200de870086}{\-Czy\-\_\-jest} (string adres)
\begin{DoxyCompactList}\small\item\em \-Metoda sprawdzająca istnienie elementu. \end{DoxyCompactList}\item 
bool \hyperlink{class_tablica_asocjacyjna_addfeb9ac2475f95e8c43eed5dfad8866}{\-Dodaj\-Element} (string adres, \-T dane)
\begin{DoxyCompactList}\small\item\em \-Metoda dodająca elementy do tablicy. \end{DoxyCompactList}\item 
bool \hyperlink{class_tablica_asocjacyjna_afdb88bfafe2bfb4171dd5ecab799b3b3}{\-Usun\-Element} (string adres)
\begin{DoxyCompactList}\small\item\em \-Metoda usuwająca element z tablicy. \end{DoxyCompactList}\item 
bool \hyperlink{class_tablica_asocjacyjna_a643d1c054d657c21db830dfe1b5232ab}{\-Zamien} (int pierwszy, int drugi)
\begin{DoxyCompactList}\small\item\em \-Metoda zamieniająca elementy w tablicy. \end{DoxyCompactList}\item 
bool \hyperlink{class_tablica_asocjacyjna_a689a3a17ff3a0595ffdefb38b7a2575f}{\-Wyszukaj} (string szukany, int lewy, int \&piwot, int prawy)
\begin{DoxyCompactList}\small\item\em \-Metoda wyszukująca elementy w tablicy. \end{DoxyCompactList}\item 
string \hyperlink{class_tablica_asocjacyjna_a9aeec8d0066efa5efce02ef4317d3011}{\-Pobierz\-Adres} (long index)
\begin{DoxyCompactList}\small\item\em \-Metoda pobierająca adres. \end{DoxyCompactList}\item 
\-T \& \hyperlink{class_tablica_asocjacyjna_a6d053fdbf4d8bbabbb2f487de8cf261d}{operator\mbox{[}$\,$\mbox{]}} (string adres)
\begin{DoxyCompactList}\small\item\em \-Przeciążenie operatora \mbox{[}\mbox{]}. \end{DoxyCompactList}\item 
\-T \& \hyperlink{class_tablica_asocjacyjna_a21688064e4bc890b91f24dbebdab6b9d}{operator\mbox{[}$\,$\mbox{]}} (long index)
\begin{DoxyCompactList}\small\item\em \-Przeciążenie operatora \mbox{[}\mbox{]}. \end{DoxyCompactList}\item 
void \hyperlink{class_tablica_asocjacyjna_af08bb1f3933f46688a62f3ad77a08bbd}{sort\-\_\-quicksort} (int left, int right)
\begin{DoxyCompactList}\small\item\em \-Realizuje sortowanie sybkie. \end{DoxyCompactList}\end{DoxyCompactItemize}


\subsection{\-Opis szczegółowy}
\subsubsection*{template$<$typename \-T$>$class Tablica\-Asocjacyjna$<$ T $>$}

\-Klasa ta zawiera wewnętrzną strukturę \-\_\-\-Dane, która odpowiada za przechowywanie adresu oraz zawartości określonego pola w tablicy oraz vector, przechowujący obiekty struktury \-Dane. \-Dodatkowo w klasie znajduje się pole typu int -\/ miejsce -\/ na potrzeby wyszukiwania binarnego. 

\-Definicja w linii 21 pliku tablica\-\_\-asocjacyjna.\-hh.



\subsection{\-Dokumentacja funkcji składowych}
\hypertarget{class_tablica_asocjacyjna_a21bae8b633b7bde70eab9200de870086}{\index{\-Tablica\-Asocjacyjna@{\-Tablica\-Asocjacyjna}!\-Czy\-\_\-jest@{\-Czy\-\_\-jest}}
\index{\-Czy\-\_\-jest@{\-Czy\-\_\-jest}!TablicaAsocjacyjna@{\-Tablica\-Asocjacyjna}}
\subsubsection[{\-Czy\-\_\-jest}]{\setlength{\rightskip}{0pt plus 5cm}template$<$typename T $>$ bool {\bf \-Tablica\-Asocjacyjna}$<$ \-T $>$\-::{\bf \-Czy\-\_\-jest} (
\begin{DoxyParamCaption}
\item[{string}]{adres}
\end{DoxyParamCaption}
)}}\label{class_tablica_asocjacyjna_a21bae8b633b7bde70eab9200de870086}
\-Metoda ta ma za zadanie sprawdzić czy w tablicy jest już zawarty element o podanym adresie.


\begin{DoxyParams}[1]{\-Parametry}
\mbox{\tt in}  & {\em adres} & -\/ adres, którego istnienie w tablicy ma sprawdzić funkcja.\\
\hline
\end{DoxyParams}
\begin{DoxyReturn}{\-Zwraca}
true -\/ gdy element o podanym adresie znajduje się już w tablicy. false -\/ gdy elementu o podanym adresie nie ma w tablicy. 
\end{DoxyReturn}


\-Definicja w linii 163 pliku tablica\-\_\-asocjacyjna.\-hh.

\hypertarget{class_tablica_asocjacyjna_addfeb9ac2475f95e8c43eed5dfad8866}{\index{\-Tablica\-Asocjacyjna@{\-Tablica\-Asocjacyjna}!\-Dodaj\-Element@{\-Dodaj\-Element}}
\index{\-Dodaj\-Element@{\-Dodaj\-Element}!TablicaAsocjacyjna@{\-Tablica\-Asocjacyjna}}
\subsubsection[{\-Dodaj\-Element}]{\setlength{\rightskip}{0pt plus 5cm}template$<$typename T $>$ bool {\bf \-Tablica\-Asocjacyjna}$<$ \-T $>$\-::{\bf \-Dodaj\-Element} (
\begin{DoxyParamCaption}
\item[{string}]{adres, }
\item[{\-T}]{dane}
\end{DoxyParamCaption}
)}}\label{class_tablica_asocjacyjna_addfeb9ac2475f95e8c43eed5dfad8866}
\-Metoda ta ma za zadanie dodać zadaną daną pod określonym adresem do tablicy.


\begin{DoxyParams}[1]{\-Parametry}
\mbox{\tt in}  & {\em adres} & -\/ adres, pod którym ma zostać umieszczona dana. \\
\hline
\mbox{\tt in}  & {\em dane} & -\/ wartość, która ma zostać umieszczona w tablicy.\\
\hline
\end{DoxyParams}
\begin{DoxyReturn}{\-Zwraca}
true -\/ gdy element został dodany. false -\/ gdy elementu nie został dodany. 
\end{DoxyReturn}


\-Definicja w linii 174 pliku tablica\-\_\-asocjacyjna.\-hh.

\hypertarget{class_tablica_asocjacyjna_a6d053fdbf4d8bbabbb2f487de8cf261d}{\index{\-Tablica\-Asocjacyjna@{\-Tablica\-Asocjacyjna}!operator\mbox{[}$\,$\mbox{]}@{operator[]}}
\index{operator\mbox{[}$\,$\mbox{]}@{operator[]}!TablicaAsocjacyjna@{\-Tablica\-Asocjacyjna}}
\subsubsection[{operator[]}]{\setlength{\rightskip}{0pt plus 5cm}template$<$typename T $>$ \-T \& {\bf \-Tablica\-Asocjacyjna}$<$ \-T $>$\-::operator\mbox{[}$\,$\mbox{]} (
\begin{DoxyParamCaption}
\item[{string}]{adres}
\end{DoxyParamCaption}
)}}\label{class_tablica_asocjacyjna_a6d053fdbf4d8bbabbb2f487de8cf261d}
\-Przeciążenie tego operatora zwraca wartość pola o podanym adresie.


\begin{DoxyParams}[1]{\-Parametry}
\mbox{\tt in}  & {\em adres} & -\/ adres pola, które ma zostać zwrócone. \\
\hline
\end{DoxyParams}
\begin{DoxyReturn}{\-Zwraca}
dane -\/ wartość danego pola. 
\end{DoxyReturn}


\-Definicja w linii 247 pliku tablica\-\_\-asocjacyjna.\-hh.

\hypertarget{class_tablica_asocjacyjna_a21688064e4bc890b91f24dbebdab6b9d}{\index{\-Tablica\-Asocjacyjna@{\-Tablica\-Asocjacyjna}!operator\mbox{[}$\,$\mbox{]}@{operator[]}}
\index{operator\mbox{[}$\,$\mbox{]}@{operator[]}!TablicaAsocjacyjna@{\-Tablica\-Asocjacyjna}}
\subsubsection[{operator[]}]{\setlength{\rightskip}{0pt plus 5cm}template$<$typename T $>$ \-T \& {\bf \-Tablica\-Asocjacyjna}$<$ \-T $>$\-::operator\mbox{[}$\,$\mbox{]} (
\begin{DoxyParamCaption}
\item[{long}]{index}
\end{DoxyParamCaption}
)}}\label{class_tablica_asocjacyjna_a21688064e4bc890b91f24dbebdab6b9d}
\-Przeciążenie tego operatora zwraca wartość pola o podanym indeksie.


\begin{DoxyParams}[1]{\-Parametry}
\mbox{\tt in}  & {\em index} & -\/ indeks pola, które ma zostać zwrócone. \\
\hline
\end{DoxyParams}
\begin{DoxyReturn}{\-Zwraca}
dane -\/ wartość danego pola. 
\end{DoxyReturn}


\-Definicja w linii 263 pliku tablica\-\_\-asocjacyjna.\-hh.

\hypertarget{class_tablica_asocjacyjna_a9aeec8d0066efa5efce02ef4317d3011}{\index{\-Tablica\-Asocjacyjna@{\-Tablica\-Asocjacyjna}!\-Pobierz\-Adres@{\-Pobierz\-Adres}}
\index{\-Pobierz\-Adres@{\-Pobierz\-Adres}!TablicaAsocjacyjna@{\-Tablica\-Asocjacyjna}}
\subsubsection[{\-Pobierz\-Adres}]{\setlength{\rightskip}{0pt plus 5cm}template$<$typename T $>$ string {\bf \-Tablica\-Asocjacyjna}$<$ \-T $>$\-::{\bf \-Pobierz\-Adres} (
\begin{DoxyParamCaption}
\item[{long}]{index}
\end{DoxyParamCaption}
)}}\label{class_tablica_asocjacyjna_a9aeec8d0066efa5efce02ef4317d3011}
\-Metoda ta pobiera adres w postaci string z pola o zadanym indeksie.


\begin{DoxyParams}[1]{\-Parametry}
\mbox{\tt in}  & {\em index} & -\/ indeks liczbowy danego pola. \\
\hline
\end{DoxyParams}
\begin{DoxyReturn}{\-Zwraca}
adres pola. 
\end{DoxyReturn}


\-Definicja w linii 238 pliku tablica\-\_\-asocjacyjna.\-hh.

\hypertarget{class_tablica_asocjacyjna_aa73ffd094410a190b69b0dbe294f0489}{\index{\-Tablica\-Asocjacyjna@{\-Tablica\-Asocjacyjna}!\-Rozmiar@{\-Rozmiar}}
\index{\-Rozmiar@{\-Rozmiar}!TablicaAsocjacyjna@{\-Tablica\-Asocjacyjna}}
\subsubsection[{\-Rozmiar}]{\setlength{\rightskip}{0pt plus 5cm}template$<$typename T $>$ long {\bf \-Tablica\-Asocjacyjna}$<$ \-T $>$\-::{\bf \-Rozmiar} (
\begin{DoxyParamCaption}
{}
\end{DoxyParamCaption}
)}}\label{class_tablica_asocjacyjna_aa73ffd094410a190b69b0dbe294f0489}


\-Definicja w linii 157 pliku tablica\-\_\-asocjacyjna.\-hh.

\hypertarget{class_tablica_asocjacyjna_af08bb1f3933f46688a62f3ad77a08bbd}{\index{\-Tablica\-Asocjacyjna@{\-Tablica\-Asocjacyjna}!sort\-\_\-quicksort@{sort\-\_\-quicksort}}
\index{sort\-\_\-quicksort@{sort\-\_\-quicksort}!TablicaAsocjacyjna@{\-Tablica\-Asocjacyjna}}
\subsubsection[{sort\-\_\-quicksort}]{\setlength{\rightskip}{0pt plus 5cm}template$<$typename T $>$ void {\bf \-Tablica\-Asocjacyjna}$<$ \-T $>$\-::{\bf sort\-\_\-quicksort} (
\begin{DoxyParamCaption}
\item[{int}]{left, }
\item[{int}]{right}
\end{DoxyParamCaption}
)}}\label{class_tablica_asocjacyjna_af08bb1f3933f46688a62f3ad77a08bbd}
\-Funkcja ta ma za zadanie posortowac wektor metoda quicksort.


\begin{DoxyParams}[1]{\-Parametry}
\mbox{\tt in}  & {\em left} & -\/ zmienna typu int okresleslajaca poczatkowe pole sortowanej tablicy. \\
\hline
\mbox{\tt in}  & {\em right} & -\/ zmienna typu int okresleslajaca koncowe pole sortowanej tablicy. \\
\hline
\end{DoxyParams}


\-Definicja w linii 291 pliku tablica\-\_\-asocjacyjna.\-hh.

\hypertarget{class_tablica_asocjacyjna_afdb88bfafe2bfb4171dd5ecab799b3b3}{\index{\-Tablica\-Asocjacyjna@{\-Tablica\-Asocjacyjna}!\-Usun\-Element@{\-Usun\-Element}}
\index{\-Usun\-Element@{\-Usun\-Element}!TablicaAsocjacyjna@{\-Tablica\-Asocjacyjna}}
\subsubsection[{\-Usun\-Element}]{\setlength{\rightskip}{0pt plus 5cm}template$<$typename T $>$ bool {\bf \-Tablica\-Asocjacyjna}$<$ \-T $>$\-::{\bf \-Usun\-Element} (
\begin{DoxyParamCaption}
\item[{string}]{adres}
\end{DoxyParamCaption}
)}}\label{class_tablica_asocjacyjna_afdb88bfafe2bfb4171dd5ecab799b3b3}
\-Metoda ta ma za zadanie usunąć element o zadanym adresie z tablicy. \-W przypadku gdy element o takiej nazwie już istnieje zostaje on podmieniony na nowy.


\begin{DoxyParams}[1]{\-Parametry}
\mbox{\tt in}  & {\em adres} & -\/ adres komórki, która ma zostać usunięta.\\
\hline
\end{DoxyParams}
\begin{DoxyReturn}{\-Zwraca}
true -\/ gdy element został podmieniony. false -\/ gdy elementu nie nie został usunięty. 
\end{DoxyReturn}


\-Definicja w linii 195 pliku tablica\-\_\-asocjacyjna.\-hh.

\hypertarget{class_tablica_asocjacyjna_a689a3a17ff3a0595ffdefb38b7a2575f}{\index{\-Tablica\-Asocjacyjna@{\-Tablica\-Asocjacyjna}!\-Wyszukaj@{\-Wyszukaj}}
\index{\-Wyszukaj@{\-Wyszukaj}!TablicaAsocjacyjna@{\-Tablica\-Asocjacyjna}}
\subsubsection[{\-Wyszukaj}]{\setlength{\rightskip}{0pt plus 5cm}template$<$typename T $>$ bool {\bf \-Tablica\-Asocjacyjna}$<$ \-T $>$\-::{\bf \-Wyszukaj} (
\begin{DoxyParamCaption}
\item[{string}]{szukany, }
\item[{int}]{lewy, }
\item[{int \&}]{piwot, }
\item[{int}]{prawy}
\end{DoxyParamCaption}
)}}\label{class_tablica_asocjacyjna_a689a3a17ff3a0595ffdefb38b7a2575f}
\-Metoda ta realizuje wyszukiwanie binarne w posortowanej tablicy. \-Efekt jej działania, czyli numer pola, w którym znajduje się szukany element, zapisywany jest w polu \char`\"{}miejsce\char`\"{} w klasie \hyperlink{class_tablica_asocjacyjna}{\-Tablica\-Asocjacyjna}.


\begin{DoxyParams}[1]{\-Parametry}
\mbox{\tt in}  & {\em szukany} & -\/ adres szukanego pola. \\
\hline
\mbox{\tt in}  & {\em lewy} & -\/ lewa granica przeszukiwanej tablicy. \\
\hline
\mbox{\tt in}  & {\em piwot} & -\/ element dzielący tablicę. \\
\hline
\mbox{\tt in}  & {\em prawy} & -\/ prawa granica przeszukiwanej tablicy. \\
\hline
\end{DoxyParams}
\begin{DoxyReturn}{\-Zwraca}
true -\/ gdy element został znaleziony. false -\/ gdy elementy nie został znaleziony. 
\end{DoxyReturn}


\-Definicja w linii 211 pliku tablica\-\_\-asocjacyjna.\-hh.

\hypertarget{class_tablica_asocjacyjna_a643d1c054d657c21db830dfe1b5232ab}{\index{\-Tablica\-Asocjacyjna@{\-Tablica\-Asocjacyjna}!\-Zamien@{\-Zamien}}
\index{\-Zamien@{\-Zamien}!TablicaAsocjacyjna@{\-Tablica\-Asocjacyjna}}
\subsubsection[{\-Zamien}]{\setlength{\rightskip}{0pt plus 5cm}template$<$typename T $>$ bool {\bf \-Tablica\-Asocjacyjna}$<$ \-T $>$\-::{\bf \-Zamien} (
\begin{DoxyParamCaption}
\item[{int}]{pierwszy, }
\item[{int}]{drugi}
\end{DoxyParamCaption}
)}}\label{class_tablica_asocjacyjna_a643d1c054d657c21db830dfe1b5232ab}
\-Metoda ta ma za zadanie zamienić miejscami dwa podane elementy tablicy.


\begin{DoxyParams}[1]{\-Parametry}
\mbox{\tt in}  & {\em pierwszy} & -\/ numer pola tablicy zajmowanego przez pierwszy z elementów. \\
\hline
\mbox{\tt in}  & {\em drugi} & -\/ numer pola tablicy zajmowanego przez drugi z elementów. \\
\hline
\end{DoxyParams}
\begin{DoxyReturn}{\-Zwraca}
true -\/ gdy elementy zostały zamienione miejscami. false -\/ gdy elementy nie zostały zamienione miejscami. 
\end{DoxyReturn}


\-Definicja w linii 274 pliku tablica\-\_\-asocjacyjna.\-hh.



\-Dokumentacja dla tej klasy została wygenerowana z pliku\-:\begin{DoxyCompactItemize}
\item 
\hyperlink{tablica__asocjacyjna_8hh}{tablica\-\_\-asocjacyjna.\-hh}\end{DoxyCompactItemize}

\hypertarget{class_tablica_asocjacyjna__db}{\section{\-Dokumentacja klasy \-Tablica\-Asocjacyjna\-\_\-db}
\label{class_tablica_asocjacyjna__db}\index{\-Tablica\-Asocjacyjna\-\_\-db@{\-Tablica\-Asocjacyjna\-\_\-db}}
}


{\ttfamily \#include $<$tablica\-\_\-asocjacyjna\-\_\-db.\-hh$>$}



\-Diagram współpracy dla \-Tablica\-Asocjacyjna\-\_\-db\-:
\subsection*{\-Metody publiczne}
\begin{DoxyCompactItemize}
\item 
int \hyperlink{class_tablica_asocjacyjna__db_adbb610341c19c92e7f31d93020407f78}{\-Rozmiar} ()
\begin{DoxyCompactList}\small\item\em \-Metoda zwracająca informację o ilości danych w tablicy. \end{DoxyCompactList}\item 
bool \hyperlink{class_tablica_asocjacyjna__db_abdb71c0bef53904ba3538cfafeb1f8a0}{\-Czy\-\_\-jest} (string adres)
\begin{DoxyCompactList}\small\item\em \-Metoda sprawdzająca istnienie elementu. \end{DoxyCompactList}\item 
bool \hyperlink{class_tablica_asocjacyjna__db_a7d3ecb28ec20bde2f72e94e104c40779}{\-Dodaj\-Element} (string adres\-\_\-dod, int dane)
\begin{DoxyCompactList}\small\item\em \-Metoda dodająca elementy do tablicy. \end{DoxyCompactList}\item 
bool \hyperlink{class_tablica_asocjacyjna__db_a7546f49a47117ad38adcf063f4fed305}{\-Usun\-Element} (string adres)
\begin{DoxyCompactList}\small\item\em \-Metoda usuwająca element z tablicy. \end{DoxyCompactList}\end{DoxyCompactItemize}
\subsection*{\-Atrybuty publiczne}
\begin{DoxyCompactItemize}
\item 
\hyperlink{class_drzewo__binarne}{\-Drzewo\-\_\-binarne} \hyperlink{class_tablica_asocjacyjna__db_acbff6baa867f8ccd15b2106e0f557c95}{drzewo}
\end{DoxyCompactItemize}


\subsection{\-Opis szczegółowy}


\-Definicja w linii 4 pliku tablica\-\_\-asocjacyjna\-\_\-db.\-hh.



\subsection{\-Dokumentacja funkcji składowych}
\hypertarget{class_tablica_asocjacyjna__db_abdb71c0bef53904ba3538cfafeb1f8a0}{\index{\-Tablica\-Asocjacyjna\-\_\-db@{\-Tablica\-Asocjacyjna\-\_\-db}!\-Czy\-\_\-jest@{\-Czy\-\_\-jest}}
\index{\-Czy\-\_\-jest@{\-Czy\-\_\-jest}!TablicaAsocjacyjna_db@{\-Tablica\-Asocjacyjna\-\_\-db}}
\subsubsection[{\-Czy\-\_\-jest}]{\setlength{\rightskip}{0pt plus 5cm}bool {\bf \-Tablica\-Asocjacyjna\-\_\-db\-::\-Czy\-\_\-jest} (
\begin{DoxyParamCaption}
\item[{string}]{adres}
\end{DoxyParamCaption}
)}}\label{class_tablica_asocjacyjna__db_abdb71c0bef53904ba3538cfafeb1f8a0}
\-Metoda ta ma za zadanie sprawdzić czy w tablicy jest już zawarty element o podanym adresie.


\begin{DoxyParams}[1]{\-Parametry}
\mbox{\tt in}  & {\em adres} & -\/ adres, którego istnienie w tablicy ma sprawdzić funkcja.\\
\hline
\end{DoxyParams}
\begin{DoxyReturn}{\-Zwraca}
true -\/ gdy element o podanym adresie znajduje się już w tablicy. false -\/ gdy elementu o podanym adresie nie ma w tablicy. 
\end{DoxyReturn}
\hypertarget{class_tablica_asocjacyjna__db_a7d3ecb28ec20bde2f72e94e104c40779}{\index{\-Tablica\-Asocjacyjna\-\_\-db@{\-Tablica\-Asocjacyjna\-\_\-db}!\-Dodaj\-Element@{\-Dodaj\-Element}}
\index{\-Dodaj\-Element@{\-Dodaj\-Element}!TablicaAsocjacyjna_db@{\-Tablica\-Asocjacyjna\-\_\-db}}
\subsubsection[{\-Dodaj\-Element}]{\setlength{\rightskip}{0pt plus 5cm}bool {\bf \-Tablica\-Asocjacyjna\-\_\-db\-::\-Dodaj\-Element} (
\begin{DoxyParamCaption}
\item[{string}]{adres\-\_\-dod, }
\item[{int}]{dane}
\end{DoxyParamCaption}
)}}\label{class_tablica_asocjacyjna__db_a7d3ecb28ec20bde2f72e94e104c40779}
\-Metoda ta ma za zadanie dodać zadaną daną pod określonym adresem do tablicy.


\begin{DoxyParams}[1]{\-Parametry}
\mbox{\tt in}  & {\em adres} & -\/ adres, pod którym ma zostać umieszczona dana. \\
\hline
\mbox{\tt in}  & {\em dane} & -\/ wartość, która ma zostać umieszczona w tablicy.\\
\hline
\end{DoxyParams}
\begin{DoxyReturn}{\-Zwraca}
true -\/ gdy element został dodany. false -\/ gdy elementu nie został dodany. 
\end{DoxyReturn}
\hypertarget{class_tablica_asocjacyjna__db_adbb610341c19c92e7f31d93020407f78}{\index{\-Tablica\-Asocjacyjna\-\_\-db@{\-Tablica\-Asocjacyjna\-\_\-db}!\-Rozmiar@{\-Rozmiar}}
\index{\-Rozmiar@{\-Rozmiar}!TablicaAsocjacyjna_db@{\-Tablica\-Asocjacyjna\-\_\-db}}
\subsubsection[{\-Rozmiar}]{\setlength{\rightskip}{0pt plus 5cm}int {\bf \-Tablica\-Asocjacyjna\-\_\-db\-::\-Rozmiar} (
\begin{DoxyParamCaption}
{}
\end{DoxyParamCaption}
)}}\label{class_tablica_asocjacyjna__db_adbb610341c19c92e7f31d93020407f78}
\hypertarget{class_tablica_asocjacyjna__db_a7546f49a47117ad38adcf063f4fed305}{\index{\-Tablica\-Asocjacyjna\-\_\-db@{\-Tablica\-Asocjacyjna\-\_\-db}!\-Usun\-Element@{\-Usun\-Element}}
\index{\-Usun\-Element@{\-Usun\-Element}!TablicaAsocjacyjna_db@{\-Tablica\-Asocjacyjna\-\_\-db}}
\subsubsection[{\-Usun\-Element}]{\setlength{\rightskip}{0pt plus 5cm}bool {\bf \-Tablica\-Asocjacyjna\-\_\-db\-::\-Usun\-Element} (
\begin{DoxyParamCaption}
\item[{string}]{adres}
\end{DoxyParamCaption}
)}}\label{class_tablica_asocjacyjna__db_a7546f49a47117ad38adcf063f4fed305}
\-Metoda ta ma za zadanie usunąć element o zadanym adresie z tablicy. \-W przypadku gdy element o takiej nazwie już istnieje zostaje on podmieniony na nowy.


\begin{DoxyParams}[1]{\-Parametry}
\mbox{\tt in}  & {\em adres} & -\/ adres komórki, która ma zostać usunięta.\\
\hline
\end{DoxyParams}
\begin{DoxyReturn}{\-Zwraca}
true -\/ gdy element został podmieniony. false -\/ gdy elementu nie nie został usunięty. 
\end{DoxyReturn}


\subsection{\-Dokumentacja atrybutów składowych}
\hypertarget{class_tablica_asocjacyjna__db_acbff6baa867f8ccd15b2106e0f557c95}{\index{\-Tablica\-Asocjacyjna\-\_\-db@{\-Tablica\-Asocjacyjna\-\_\-db}!drzewo@{drzewo}}
\index{drzewo@{drzewo}!TablicaAsocjacyjna_db@{\-Tablica\-Asocjacyjna\-\_\-db}}
\subsubsection[{drzewo}]{\setlength{\rightskip}{0pt plus 5cm}{\bf \-Drzewo\-\_\-binarne} {\bf \-Tablica\-Asocjacyjna\-\_\-db\-::drzewo}}}\label{class_tablica_asocjacyjna__db_acbff6baa867f8ccd15b2106e0f557c95}


\-Definicja w linii 7 pliku tablica\-\_\-asocjacyjna\-\_\-db.\-hh.



\-Dokumentacja dla tej klasy została wygenerowana z pliku\-:\begin{DoxyCompactItemize}
\item 
\hyperlink{tablica__asocjacyjna__db_8hh}{tablica\-\_\-asocjacyjna\-\_\-db.\-hh}\end{DoxyCompactItemize}

\chapter{\-Dokumentacja plików}
\hypertarget{drzewo__binarne_8cpp}{\section{\-Dokumentacja pliku drzewo\-\_\-binarne.\-cpp}
\label{drzewo__binarne_8cpp}\index{drzewo\-\_\-binarne.\-cpp@{drzewo\-\_\-binarne.\-cpp}}
}


\-Definicje metod klasy \hyperlink{class_drzewo__binarne}{\-Drzewo\-\_\-binarne}.  


{\ttfamily \#include \char`\"{}drzewo\-\_\-binarne.\-hh\char`\"{}}\*
\-Wykres zależności załączania dla drzewo\-\_\-binarne.\-cpp\-:


\subsection{\-Opis szczegółowy}


\-Definicja w pliku \hyperlink{drzewo__binarne_8cpp_source}{drzewo\-\_\-binarne.\-cpp}.


\hypertarget{drzewo__binarne_8hh}{\section{\-Dokumentacja pliku drzewo\-\_\-binarne.\-hh}
\label{drzewo__binarne_8hh}\index{drzewo\-\_\-binarne.\-hh@{drzewo\-\_\-binarne.\-hh}}
}


\-Zawiera definicje klasy \hyperlink{class_drzewo__binarne}{\-Drzewo\-\_\-binarne}, jej metody oraz polecenia załączenia niezbędnych bibliotek.  


{\ttfamily \#include $<$iostream$>$}\*
{\ttfamily \#include $<$stdlib.\-h$>$}\*
\-Wykres zależności załączania dla drzewo\-\_\-binarne.\-hh\-:
\-Ten wykres pokazuje, które pliki bezpośrednio lub pośrednio załączają ten plik\-:
\subsection*{\-Komponenty}
\begin{DoxyCompactItemize}
\item 
class \hyperlink{class_drzewo__binarne}{\-Drzewo\-\_\-binarne}
\begin{DoxyCompactList}\small\item\em \-Klasa definiująca drzewo binarne realizujące implementację tablicy asocjacyjnej. \end{DoxyCompactList}\item 
struct \hyperlink{struct_drzewo__binarne_1_1galaz}{\-Drzewo\-\_\-binarne\-::galaz}
\begin{DoxyCompactList}\small\item\em \-Struktura realizująca \char`\"{}gałęzie\char`\"{} drzewa binarnego. \end{DoxyCompactList}\end{DoxyCompactItemize}


\subsection{\-Opis szczegółowy}


\-Definicja w pliku \hyperlink{drzewo__binarne_8hh_source}{drzewo\-\_\-binarne.\-hh}.


\hypertarget{funkcje_8cpp}{\section{\-Dokumentacja pliku funkcje.\-cpp}
\label{funkcje_8cpp}\index{funkcje.\-cpp@{funkcje.\-cpp}}
}


\-Zawiera definicje funkcji uzytych w programie.  


{\ttfamily \#include \char`\"{}sortowanie.\-hh\char`\"{}}\*
\-Wykres zależności załączania dla funkcje.\-cpp\-:
\subsection*{\-Funkcje}
\begin{DoxyCompactItemize}
\item 
istream \& \hyperlink{funkcje_8cpp_a399582d6055e12e20bbd4e3b7c96e0d1}{operator$>$$>$} (istream \&in, \hyperlink{class_dane}{\-Dane} \&wekt)
\item 
ostream \& \hyperlink{funkcje_8cpp_a84cb63c4164f3189c8fefd91e9c21135}{operator$<$$<$} (ostream \&out, const \hyperlink{class_dane}{\-Dane} \&wekt)
\item 
void \hyperlink{funkcje_8cpp_a26b47c1c0a853ec6e39c299da7eff8b3}{\-Zapis} (double czas\-\_\-sredni, int ilosc\-\_\-powt, int ilosc\-\_\-elementow)
\begin{DoxyCompactList}\small\item\em \-Funkcja zapisujaca do pliku. \end{DoxyCompactList}\item 
double \hyperlink{funkcje_8cpp_ac1e549037ac5915fb1018a206697594d}{\-Uruchom} (int ilosc\-\_\-powtorzen, \hyperlink{class_dane}{\-Dane} wektor1, \hyperlink{class_dane}{\-Dane} wektor2)
\begin{DoxyCompactList}\small\item\em \-Funkcja uruchamiajaca algorytm. \end{DoxyCompactList}\end{DoxyCompactItemize}


\subsection{\-Opis szczegółowy}


\-Definicja w pliku \hyperlink{funkcje_8cpp_source}{funkcje.\-cpp}.



\subsection{\-Dokumentacja funkcji}
\hypertarget{funkcje_8cpp_a84cb63c4164f3189c8fefd91e9c21135}{\index{funkcje.\-cpp@{funkcje.\-cpp}!operator$<$$<$@{operator$<$$<$}}
\index{operator$<$$<$@{operator$<$$<$}!funkcje.cpp@{funkcje.\-cpp}}
\subsubsection[{operator$<$$<$}]{\setlength{\rightskip}{0pt plus 5cm}ostream\& operator$<$$<$ (
\begin{DoxyParamCaption}
\item[{ostream \&}]{out, }
\item[{const {\bf \-Dane} \&}]{wekt}
\end{DoxyParamCaption}
)}}\label{funkcje_8cpp_a84cb63c4164f3189c8fefd91e9c21135}
\-Przeciazony operator pozwalajacy na wypisywanie wartosci pol obiektow klasy \hyperlink{class_dane}{\-Dane}.

\-Parametry\-:


\begin{DoxyParams}[1]{\-Parametry}
\mbox{\tt out}  & {\em out} & -\/ \-Przekazywany przez referencje strumien danych wyjsciowych. \\
\hline
\mbox{\tt out}  & {\em wekt} & -\/ \-Obiekt klasy \hyperlink{class_dane}{\-Dane}, przekazywany przez referencje, ktorego pola maja zostac wypisane. \\
\hline
\end{DoxyParams}


\-Definicja w linii 25 pliku funkcje.\-cpp.

\hypertarget{funkcje_8cpp_a399582d6055e12e20bbd4e3b7c96e0d1}{\index{funkcje.\-cpp@{funkcje.\-cpp}!operator$>$$>$@{operator$>$$>$}}
\index{operator$>$$>$@{operator$>$$>$}!funkcje.cpp@{funkcje.\-cpp}}
\subsubsection[{operator$>$$>$}]{\setlength{\rightskip}{0pt plus 5cm}istream\& operator$>$$>$ (
\begin{DoxyParamCaption}
\item[{istream \&}]{in, }
\item[{{\bf \-Dane} \&}]{wekt}
\end{DoxyParamCaption}
)}}\label{funkcje_8cpp_a399582d6055e12e20bbd4e3b7c96e0d1}
\-Przeciazony operator pozwalajacy na wpisywanie wartosci do pol obiektow klasy \hyperlink{class_dane}{\-Dane}.

\-Parametry\-:


\begin{DoxyParams}[1]{\-Parametry}
\mbox{\tt in}  & {\em in} & -\/ \-Przekazywany przez referencje strumien danych wejsciowych. \\
\hline
\mbox{\tt out}  & {\em wekt} & -\/ \-Obiekt klasy \hyperlink{class_dane}{\-Dane}, ktorego pola maja zostac uzupelnione danymi wjesciowymi. \\
\hline
\end{DoxyParams}


\-Definicja w linii 10 pliku funkcje.\-cpp.

\hypertarget{funkcje_8cpp_ac1e549037ac5915fb1018a206697594d}{\index{funkcje.\-cpp@{funkcje.\-cpp}!\-Uruchom@{\-Uruchom}}
\index{\-Uruchom@{\-Uruchom}!funkcje.cpp@{funkcje.\-cpp}}
\subsubsection[{\-Uruchom}]{\setlength{\rightskip}{0pt plus 5cm}double {\bf \-Uruchom} (
\begin{DoxyParamCaption}
\item[{int}]{ilosc\-\_\-powtorzen, }
\item[{{\bf \-Dane}}]{wektor1, }
\item[{{\bf \-Dane}}]{wektor2}
\end{DoxyParamCaption}
)}}\label{funkcje_8cpp_ac1e549037ac5915fb1018a206697594d}
\-Funkcja majaca na celu uruchomienie algorytmu zadana ilosc razy, zmierzenie czasu jego dzialania i porownanie wyniku z plikiem wzorcowym.

\-Parametry\-:


\begin{DoxyParams}[1]{\-Parametry}
\mbox{\tt in}  & {\em ilosc\-\_\-powtorzen} & -\/ \-Ilosc powtorzen wykonania algorytmu. \\
\hline
\mbox{\tt in}  & {\em wektor1} & -\/ \-Obiekt klasy \hyperlink{class_dane}{\-Dane} zawierajacy dane, na ktorych maja byc wykonane operacje przez algorytm. \\
\hline
\mbox{\tt in}  & {\em wektor2} & -\/ \-Obiekt klasy \hyperlink{class_dane}{\-Dane} zawierajacy dane pobrane z pliku wzorcowego, w celu porownania wynikow. \\
\hline
\end{DoxyParams}
\begin{DoxyReturn}{\-Zwraca}
czas -\/ \-Czas wykonania algorytmu. 
\end{DoxyReturn}


\-Definicja w linii 196 pliku funkcje.\-cpp.



\-Oto graf wywołań dla tej funkcji\-:




\-Oto graf wywoływań tej funkcji\-:


\hypertarget{funkcje_8cpp_a26b47c1c0a853ec6e39c299da7eff8b3}{\index{funkcje.\-cpp@{funkcje.\-cpp}!\-Zapis@{\-Zapis}}
\index{\-Zapis@{\-Zapis}!funkcje.cpp@{funkcje.\-cpp}}
\subsubsection[{\-Zapis}]{\setlength{\rightskip}{0pt plus 5cm}void {\bf \-Zapis} (
\begin{DoxyParamCaption}
\item[{double}]{czas\-\_\-sredni, }
\item[{int}]{ilosc\-\_\-powt, }
\item[{int}]{ilosc\-\_\-elementow}
\end{DoxyParamCaption}
)}}\label{funkcje_8cpp_a26b47c1c0a853ec6e39c299da7eff8b3}
\-Funkcja majaca na celu zapisanie informacji o srednim czasie wykonywania algorytmu, ilosci powtorzen jego wykonania oraz ilosci elementow, na ktorych zostala wykonana operacja do pliku csv.

\-Parametry\-:


\begin{DoxyParams}[1]{\-Parametry}
\mbox{\tt in}  & {\em czas\-\_\-sredni} & -\/ \-Sredni czas jednorazowego wykonania algorytmu. \\
\hline
\mbox{\tt in}  & {\em ilosc\-\_\-powt} & -\/ \-Ilosc powtorzen wykonania algorytmu. \\
\hline
\mbox{\tt in}  & {\em ilosc\-\_\-elementow} & -\/ ilosc elementow, na ktorych wykonano operacje. \\
\hline
\end{DoxyParams}


\-Definicja w linii 185 pliku funkcje.\-cpp.



\-Oto graf wywoływań tej funkcji\-:



\hypertarget{funkcje_8hh}{\section{\-Dokumentacja pliku funkcje.\-hh}
\label{funkcje_8hh}\index{funkcje.\-hh@{funkcje.\-hh}}
}


\-Zawiera deklaracje funkcji oraz instrukcje zalaczenia bibliotek.  


{\ttfamily \#include $<$iostream$>$}\*
{\ttfamily \#include $<$fstream$>$}\*
\subsection*{\-Funkcje}
\begin{DoxyCompactItemize}
\item 
int $\ast$ \hyperlink{funkcje_8hh_a8bbf700936be11136643474b49fa1764}{\-Wczytywanie} (const char $\ast$nazwa)
\begin{DoxyCompactList}\small\item\em \-Wczytuje dane z pliku. \end{DoxyCompactList}\item 
int $\ast$ \hyperlink{funkcje_8hh_ae11a32cee0192a595fcb71a98c32ee71}{\-Mnozenie} (int tablica\mbox{[}$\,$\mbox{]})
\begin{DoxyCompactList}\small\item\em \-Wykonuje operacje mnozenia. \end{DoxyCompactList}\item 
bool \hyperlink{funkcje_8hh_a721a30457d3f93333a2389712ab9153d}{\-Porownanie} (int wynik\mbox{[}$\,$\mbox{]}, int porownanie\mbox{[}$\,$\mbox{]})
\begin{DoxyCompactList}\small\item\em \-Porownuje dwie tablice. \end{DoxyCompactList}\end{DoxyCompactItemize}


\subsection{\-Opis szczegółowy}


\-Definicja w pliku \hyperlink{funkcje_8hh_source}{funkcje.\-hh}.



\subsection{\-Dokumentacja funkcji}
\hypertarget{funkcje_8hh_ae11a32cee0192a595fcb71a98c32ee71}{\index{funkcje.\-hh@{funkcje.\-hh}!\-Mnozenie@{\-Mnozenie}}
\index{\-Mnozenie@{\-Mnozenie}!funkcje.hh@{funkcje.\-hh}}
\subsubsection[{\-Mnozenie}]{\setlength{\rightskip}{0pt plus 5cm}int$\ast$ {\bf \-Mnozenie} (
\begin{DoxyParamCaption}
\item[{int}]{tablica\mbox{[}$\,$\mbox{]}}
\end{DoxyParamCaption}
)}}\label{funkcje_8hh_ae11a32cee0192a595fcb71a98c32ee71}
\-Funkcja ta ma za zadanie przemnozyc wszystkie elementy podanej tablicy przez 2.

\-Parametry\-:


\begin{DoxyParams}[1]{\-Parametry}
\mbox{\tt in}  & {\em tablica\mbox{[}$\,$\mbox{]}} & -\/ tablica wartosci, ktore maja zostac wymnozone \\
\hline
\end{DoxyParams}
\begin{DoxyReturn}{\-Zwraca}
\-Funkcja zwraca wskaznik na tablice zawierajaca wyniki dzialania. 
\end{DoxyReturn}


\-Definicja w linii 30 pliku funkcje.\-cpp.

\hypertarget{funkcje_8hh_a721a30457d3f93333a2389712ab9153d}{\index{funkcje.\-hh@{funkcje.\-hh}!\-Porownanie@{\-Porownanie}}
\index{\-Porownanie@{\-Porownanie}!funkcje.hh@{funkcje.\-hh}}
\subsubsection[{\-Porownanie}]{\setlength{\rightskip}{0pt plus 5cm}bool {\bf \-Porownanie} (
\begin{DoxyParamCaption}
\item[{int}]{wynik\mbox{[}$\,$\mbox{]}, }
\item[{int}]{porownanie\mbox{[}$\,$\mbox{]}}
\end{DoxyParamCaption}
)}}\label{funkcje_8hh_a721a30457d3f93333a2389712ab9153d}
\-Funkcja ta ma za zadanie porownac odpowiadajace elementy dwoch tablic o takich samych wymiarach. \-W przypadku znalezienia roznego elementu operacja zostaje przerwana.

\-Parametry\-:


\begin{DoxyParams}[1]{\-Parametry}
\mbox{\tt in}  & {\em wynik\mbox{[}$\,$\mbox{]}} & -\/ pierwsza z porownywanych tablic \\
\hline
\mbox{\tt in}  & {\em porownanie\mbox{[}$\,$\mbox{]}} & -\/ druga z porownywanych tablic \\
\hline
\end{DoxyParams}
\begin{DoxyReturn}{\-Zwraca}
\-Funkcja zwraca wartosc true, jesli tablice sa takie same oraz false, jesli posiadaja chociaz jeden rozny element. 
\end{DoxyReturn}


\-Definicja w linii 43 pliku funkcje.\-cpp.

\hypertarget{funkcje_8hh_a8bbf700936be11136643474b49fa1764}{\index{funkcje.\-hh@{funkcje.\-hh}!\-Wczytywanie@{\-Wczytywanie}}
\index{\-Wczytywanie@{\-Wczytywanie}!funkcje.hh@{funkcje.\-hh}}
\subsubsection[{\-Wczytywanie}]{\setlength{\rightskip}{0pt plus 5cm}int$\ast$ {\bf \-Wczytywanie} (
\begin{DoxyParamCaption}
\item[{const char $\ast$}]{nazwa}
\end{DoxyParamCaption}
)}}\label{funkcje_8hh_a8bbf700936be11136643474b49fa1764}
\-Funkcja ta ma za zadanie wczytac dane, w postaci cyfr, ze wskazanego pliku.

\-Parametry\-:


\begin{DoxyParams}[1]{\-Parametry}
\mbox{\tt in}  & {\em $\ast$nazwa} & -\/ nazwa wczytywanego pliku, przekazywana przez wskaznik \\
\hline
\end{DoxyParams}
\begin{DoxyReturn}{\-Zwraca}
\-Funkcja zwraca wskaznik na tablice wczytanych znakow. 
\end{DoxyReturn}


\-Definicja w linii 8 pliku funkcje.\-cpp.


\hypertarget{main_8cpp}{\section{\-Dokumentacja pliku main.\-cpp}
\label{main_8cpp}\index{main.\-cpp@{main.\-cpp}}
}


\-Zawiera definicje glownej funkcji programu.  


{\ttfamily \#include \char`\"{}funkcje.\-hh\char`\"{}}\*
\subsection*{\-Funkcje}
\begin{DoxyCompactItemize}
\item 
int \hyperlink{main_8cpp_ae66f6b31b5ad750f1fe042a706a4e3d4}{main} ()
\begin{DoxyCompactList}\small\item\em \-Wywoluje odpowiednie funkcje. \end{DoxyCompactList}\end{DoxyCompactItemize}


\subsection{\-Opis szczegółowy}


\-Definicja w pliku \hyperlink{main_8cpp_source}{main.\-cpp}.



\subsection{\-Dokumentacja funkcji}
\hypertarget{main_8cpp_ae66f6b31b5ad750f1fe042a706a4e3d4}{\index{main.\-cpp@{main.\-cpp}!main@{main}}
\index{main@{main}!main.cpp@{main.\-cpp}}
\subsubsection[{main}]{\setlength{\rightskip}{0pt plus 5cm}int {\bf main} (
\begin{DoxyParamCaption}
{}
\end{DoxyParamCaption}
)}}\label{main_8cpp_ae66f6b31b5ad750f1fe042a706a4e3d4}
\-Jest to glowna funkcja programu odpowiedzialna za wywolanie w odpowiedniej kolejnosci innych funkcji z okreslonymi parametrami. \-Odpowiada rowniez za informowanie uzytkownika o poprawnosci wyniku oraz czasie dzialania algorytmu mnozenia. 

\-Definicja w linii 40 pliku main.\-cpp.


\hypertarget{tablica__asocjacyjna_8hh}{\section{\-Dokumentacja pliku tablica\-\_\-asocjacyjna.\-hh}
\label{tablica__asocjacyjna_8hh}\index{tablica\-\_\-asocjacyjna.\-hh@{tablica\-\_\-asocjacyjna.\-hh}}
}


\-Zawiera definicję klasy \hyperlink{class_tablica_asocjacyjna}{\-Tablica\-Asocjacyjna}, jej metod oraz instrukcje załączenia poszczególnych bibliotek.  


{\ttfamily \#include $<$iostream$>$}\*
{\ttfamily \#include $<$vector$>$}\*
{\ttfamily \#include $<$stdlib.\-h$>$}\*
\-Wykres zależności załączania dla tablica\-\_\-asocjacyjna.\-hh\-:
\-Ten wykres pokazuje, które pliki bezpośrednio lub pośrednio załączają ten plik\-:
\subsection*{\-Komponenty}
\begin{DoxyCompactItemize}
\item 
class \hyperlink{class_tablica_asocjacyjna}{\-Tablica\-Asocjacyjna$<$ T $>$}
\begin{DoxyCompactList}\small\item\em \-Klasa definiujaca tablice asocjacyjna zaimplementowana przy pomocy wektora. \end{DoxyCompactList}\item 
struct {\bfseries \-Tablica\-Asocjacyjna$<$ T $>$\-::\-\_\-\-Dane}
\end{DoxyCompactItemize}


\subsection{\-Opis szczegółowy}


\-Definicja w pliku \hyperlink{tablica__asocjacyjna_8hh_source}{tablica\-\_\-asocjacyjna.\-hh}.


\hypertarget{tablica__asocjacyjna__db_8hh}{\section{\-Dokumentacja pliku tablica\-\_\-asocjacyjna\-\_\-db.\-hh}
\label{tablica__asocjacyjna__db_8hh}\index{tablica\-\_\-asocjacyjna\-\_\-db.\-hh@{tablica\-\_\-asocjacyjna\-\_\-db.\-hh}}
}
{\ttfamily \#include \char`\"{}funkcje.\-hh\char`\"{}}\*
{\ttfamily \#include \char`\"{}drzewo\-\_\-binarne.\-hh\char`\"{}}\*
\-Wykres zależności załączania dla tablica\-\_\-asocjacyjna\-\_\-db.\-hh\-:
\-Ten wykres pokazuje, które pliki bezpośrednio lub pośrednio załączają ten plik\-:
\subsection*{\-Komponenty}
\begin{DoxyCompactItemize}
\item 
class \hyperlink{class_tablica_asocjacyjna__db}{\-Tablica\-Asocjacyjna\-\_\-db}
\end{DoxyCompactItemize}

\hypertarget{tablica__haszujaca_8cpp}{\section{\-Dokumentacja pliku tablica\-\_\-haszujaca.\-cpp}
\label{tablica__haszujaca_8cpp}\index{tablica\-\_\-haszujaca.\-cpp@{tablica\-\_\-haszujaca.\-cpp}}
}


\-Zawiera definicje metod klasy \hyperlink{class_tablica__haszujaca}{\-Tablica\-\_\-haszujaca} i \hyperlink{class_element}{\-Element}.  


{\ttfamily \#include \char`\"{}tablica\-\_\-haszujaca.\-hh\char`\"{}}\*
\-Wykres zależności załączania dla tablica\-\_\-haszujaca.\-cpp\-:
\subsection*{\-Funkcje}
\begin{DoxyCompactItemize}
\item 
ostream \& \hyperlink{tablica__haszujaca_8cpp_a5fbcf8eef969e643764a485bc6f3a100}{operator$<$$<$} (ostream \&ostr, \hyperlink{class_element}{\-Element} \&elem)
\end{DoxyCompactItemize}


\subsection{\-Opis szczegółowy}


\-Definicja w pliku \hyperlink{tablica__haszujaca_8cpp_source}{tablica\-\_\-haszujaca.\-cpp}.



\subsection{\-Dokumentacja funkcji}
\hypertarget{tablica__haszujaca_8cpp_a5fbcf8eef969e643764a485bc6f3a100}{\index{tablica\-\_\-haszujaca.\-cpp@{tablica\-\_\-haszujaca.\-cpp}!operator$<$$<$@{operator$<$$<$}}
\index{operator$<$$<$@{operator$<$$<$}!tablica_haszujaca.cpp@{tablica\-\_\-haszujaca.\-cpp}}
\subsubsection[{operator$<$$<$}]{\setlength{\rightskip}{0pt plus 5cm}ostream\& operator$<$$<$ (
\begin{DoxyParamCaption}
\item[{ostream \&}]{ostr, }
\item[{{\bf \-Element} \&}]{elem}
\end{DoxyParamCaption}
)}}\label{tablica__haszujaca_8cpp_a5fbcf8eef969e643764a485bc6f3a100}
\-Ten przeładowany operator wyjścia odpowiada za podwanie na standardowe wyjście zawartości obiektu klasy \hyperlink{class_element}{\-Element}.


\begin{DoxyParams}[1]{\-Parametry}
\mbox{\tt out}  & {\em ostr} & -\/ strumień wyjściowy. \\
\hline
\mbox{\tt in}  & {\em elem} & -\/ drukowany obiekt. \\
\hline
\end{DoxyParams}


\-Definicja w linii 65 pliku tablica\-\_\-haszujaca.\-cpp.


\hypertarget{tablica__haszujaca_8hh}{\section{\-Dokumentacja pliku tablica\-\_\-haszujaca.\-hh}
\label{tablica__haszujaca_8hh}\index{tablica\-\_\-haszujaca.\-hh@{tablica\-\_\-haszujaca.\-hh}}
}


\-Zawiera definicje klase \hyperlink{class_tablica__haszujaca}{\-Tablica\-\_\-haszujaca} oraz \hyperlink{class_element}{\-Element}, ich metod oraz instrukcje załączenia poszczególnych bibliotek.  


{\ttfamily \#include $<$iostream$>$}\*
{\ttfamily \#include $<$string$>$}\*
{\ttfamily \#include $<$sstream$>$}\*
\-Wykres zależności załączania dla tablica\-\_\-haszujaca.\-hh\-:
\-Ten wykres pokazuje, które pliki bezpośrednio lub pośrednio załączają ten plik\-:
\subsection*{\-Komponenty}
\begin{DoxyCompactItemize}
\item 
class \hyperlink{class_element}{\-Element}
\begin{DoxyCompactList}\small\item\em \-Klasa definiująca element umieszczany w tablicy asocjacyjnej. \end{DoxyCompactList}\item 
class \hyperlink{class_tablica__haszujaca}{\-Tablica\-\_\-haszujaca}
\begin{DoxyCompactList}\small\item\em \-Klasa definiująca tablicę haszującą realizującą implementację tablicy asocjacyjnej. \end{DoxyCompactList}\end{DoxyCompactItemize}


\subsection{\-Opis szczegółowy}


\-Definicja w pliku \hyperlink{tablica__haszujaca_8hh_source}{tablica\-\_\-haszujaca.\-hh}.


\printindex
\end{document}
